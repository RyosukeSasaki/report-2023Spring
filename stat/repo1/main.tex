\documentclass[uplatex,a4j,11pt,dvipdfmx]{jsarticle}
\usepackage{listings,jvlisting}
\bibliographystyle{junsrt}

\usepackage{url}

\usepackage{graphicx}
\usepackage{gnuplot-lua-tikz}
\usepackage{pgfplots}
\usepackage{tikz}
\usepackage{amsmath,amsfonts,amssymb}
\usepackage{bm}
\usepackage{siunitx}

\makeatletter
\def\fgcaption{\def\@captype{figure}\caption}
\makeatother
\newcommand{\setsections}[3]{
\setcounter{section}{#1}
\setcounter{subsection}{#2}
\setcounter{subsubsection}{#3}
}
\newcommand{\mfig}[3][width=15cm]{
\begin{center}
\includegraphics[#1]{#2}
\fgcaption{#3 \label{fig:#2}}
\end{center}
}
\newcommand{\gnu}[2]{
\begin{figure}[hptb]
\begin{center}
\input{#2}
\caption{#1}
\label{fig:#2}
\end{center}
\end{figure}
}

\begin{document}
\title{統計物理学 No.1}
\author{82311971 佐々木良輔}
\date{}
\maketitle
\section*{超高密度符号化}
%超高密度符号化では以下の手順に従いAliceからBobへ情報を送信する.
%\begin{enumerate}
%  \item Bell状態にある2量子ビット(A,B)を用意し, AliceにA, BobにBを渡しておく
%  \item Aliceは送信したい2古典ビット(a,b)に応じて手元の量子ビットAを操作する
%  \item Aliceは操作されたAをBobに渡す
%  \item Bobは手元の(A,B)に適当な操作をし,測定することで(a,b)を得る
%\end{enumerate}
%ここでBobは1量子ビットAのみをAliceから受け取ったのに対して,
%2古典ビット(a,b)の情報を得ている.これが超高密度符号化である.
AliceとBobはもつれた量子ビット$(A,B)$のうちそれぞれ1つを持っている.
Aliceは送信したい2古典ビットに応じて異なる操作を量子ビット$A$に施し,これをBobに渡す.
Bobはもともと手元にあった量子ビット$B$とAliceから渡された量子ビット$A$に適切な操作を行うことで,
元の古典ビットabの情報を得ることが出来る.
ここでAliceはBobに対して1量子ビットしか送信していないが,Bobは2古典ビット分の情報を得ており,
これを超高密度符号化と呼ぶ.
超高密度符号化は具体的には以下の手順で行われる.
\begin{enumerate}
  \item Bell状態$\frac{1}{\sqrt{2}}(|00\rangle+|11\rangle)$にある量子ビット$(A,B)$をAliceとBobに渡す.
  \item Aliceは送信したい古典ビット$ab$に応じて以下の操作を$A$に施す.
  \begin{enumerate}
    \item $ab=(00)_2$のとき, Aliceは$A$に何もしない.このとき$(A,B)$は
    \begin{align}
      |{\rm Bell}\rangle=\frac{1}{\sqrt{2}}(|00\rangle+|11\rangle)
    \end{align}
    である.
    \item $ab=(01)_2$のとき, Aliceは$A$に$\sigma_x$を適用する.
    $\sigma_x$を$|0\rangle$, $|1\rangle$に適用したとき,それぞれ
    \begin{align}
      \begin{split}
        &\sigma_x|0\rangle=\begin{pmatrix}
          0&1\\1&0
        \end{pmatrix}\begin{pmatrix}
          1\\0
        \end{pmatrix}=|1\rangle,\\
        &\sigma_x|1\rangle=\begin{pmatrix}
          0&1\\1&0
        \end{pmatrix}\begin{pmatrix}
          0\\1
        \end{pmatrix}=|0\rangle
      \end{split}
    \end{align}
    となるため操作後の$(A,B)$は
    \begin{align}
      \sigma_x|{\rm Bell}\rangle=\frac{1}{\sqrt{2}}(|10\rangle+|01\rangle)
    \end{align}
    である.
    \newpage
    \item $ab=(10)_2$のとき, Aliceは$A$に$\sigma_z$を適用する.
    $\sigma_z$を$|0\rangle$, $|1\rangle$に適用したとき,それぞれ
    \begin{align}
      \begin{split}
        &\sigma_z|0\rangle=\begin{pmatrix}
          1&0\\0&-1
        \end{pmatrix}\begin{pmatrix}
          1\\0
        \end{pmatrix}=|0\rangle,\\
        &\sigma_z|1\rangle=\begin{pmatrix}
          1&0\\0&-1
        \end{pmatrix}\begin{pmatrix}
          0\\1
        \end{pmatrix}=-|1\rangle
      \end{split}
    \end{align}
    となるため操作後の$(A,B)$は
    \begin{align}
      \sigma_z|{\rm Bell}\rangle=\frac{1}{\sqrt{2}}(|00\rangle-|11\rangle)
    \end{align}
    である.
    \item $ab=(11)_2$のとき, Aliceは$A$に$\sigma_z\sigma_x$を適用する.
    $\sigma_z\sigma_x$を$|0\rangle$, $|1\rangle$に適用したとき,それぞれ
    \begin{align}
      \begin{split}
        &\sigma_z\sigma_x|0\rangle=-|1\rangle,\\
        &\sigma_z\sigma_x|1\rangle=|0\rangle
      \end{split}
    \end{align}
    となるため操作後の$(A,B)$は
    \begin{align}
      \sigma_z\sigma_x|{\rm Bell}\rangle=\frac{1}{\sqrt{2}}(-|10\rangle+|01\rangle)
    \end{align}
    である.
  \end{enumerate}
  \item Bobは2の操作を施されたAを受け取り, $(A,B)$にCNOTを適用する.
  (a)〜(d)それぞれについてCNOTを適用すると以下のようになる.
  \begin{enumerate}
    \item \begin{align}
      {\rm CNOT}\frac{1}{\sqrt{2}}(|00\rangle+|11\rangle)=\frac{1}{\sqrt{2}}(|0\rangle+|1\rangle)|0\rangle
    \end{align}
    \item \begin{align}
      {\rm CNOT}\frac{1}{\sqrt{2}}(|10\rangle+|01\rangle)=\frac{1}{\sqrt{2}}(|1\rangle+|0\rangle)|1\rangle
    \end{align}
    \item \begin{align}
      {\rm CNOT}\frac{1}{\sqrt{2}}(|00\rangle-|11\rangle)=\frac{1}{\sqrt{2}}(|0\rangle-|1\rangle)|0\rangle
    \end{align}
    \item \begin{align}
      {\rm CNOT}\frac{1}{\sqrt{2}}(-|10\rangle+|01\rangle)=\frac{1}{\sqrt{2}}(-|1\rangle+|0\rangle)|1\rangle
    \end{align}
  \end{enumerate}
  \newpage
  \item Bobは$A$にHadamard変換を行う.
  (a)〜(d)それぞれについてHadamard変換をすると以下のようになる.
  \begin{enumerate}
    \item $\frac{1}{\sqrt{2}}(|0\rangle+|1\rangle)$にHadamard変換を行うと
    \begin{align}
      \frac{1}{\sqrt{2}}\begin{pmatrix}
        1&1\\1&-1
      \end{pmatrix}
      \frac{1}{\sqrt{2}}(|0\rangle+|1\rangle)=\frac{1}{2}\begin{pmatrix}
        1&1\\1&-1
      \end{pmatrix}\begin{pmatrix}
        1\\1
      \end{pmatrix}=|0\rangle
    \end{align}
    よって$(A,B)$は
    \begin{align}
      H\frac{1}{\sqrt{2}}(|0\rangle+|1\rangle)|0\rangle=|00\rangle
    \end{align}
    である.
    \item $\frac{1}{\sqrt{2}}(|1\rangle+|0\rangle)$にHadamard変換を行うと$|0\rangle$である.
    よって$(A,B)$は
    \begin{align}
      H\frac{1}{\sqrt{2}}(|0\rangle+|1\rangle)|1\rangle=|01\rangle
    \end{align}
    である.
    \item $\frac{1}{\sqrt{2}}(|0\rangle-|1\rangle)$にHadamard変換を行うと
    \begin{align}
      \frac{1}{\sqrt{2}}\begin{pmatrix}
        1&1\\1&-1
      \end{pmatrix}
      \frac{1}{\sqrt{2}}(|0\rangle-|1\rangle)=\frac{1}{2}\begin{pmatrix}
        1&1\\1&-1
      \end{pmatrix}\begin{pmatrix}
        1\\-1
      \end{pmatrix}=|1\rangle
    \end{align}
    よって$(A,B)$は
    \begin{align}
      H\frac{1}{\sqrt{2}}(|0\rangle-|1\rangle)|0\rangle=|10\rangle
    \end{align}
    である.
    \item $\frac{1}{\sqrt{2}}(-|1\rangle+|0\rangle)$にHadamard変換を行うと$|1\rangle$である.
    よって$(A,B)$は
    \begin{align}
      H\frac{1}{\sqrt{2}}(-|1\rangle+|0\rangle)|1\rangle=|11\rangle
    \end{align}
    である.
  \end{enumerate}
  \item 以上の操作で$ab$対応する操作を行った$(A,B)$の状態は$|ab\rangle$となった.
  したがって$(A,B)$を測定することで$ab$を得る.
\end{enumerate}
以上の操作は量子回路では図\ref{fig:superdense_coding.pdf}のように表される.上で示した手順の直後の状態が図中(1)〜(4)での位置に相当する.
また$\sigma_i$は$ab$に応じて$\sigma_0$, $\sigma_x$, $\sigma_z$, $\sigma_z\sigma_x$になる.
\mfig[width=8cm]{superdense_coding.pdf}{超高密度符号化の量子回路, (1)〜(4)は上で示した各手順の後に相当する.}
\bibliography{ref.bib}
\end{document}