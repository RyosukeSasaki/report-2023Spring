\documentclass[uplatex,a4j,11pt,dvipdfmx]{jsarticle}
\usepackage{listings,jvlisting}
\bibliographystyle{junsrt}

\usepackage{url}

\usepackage{graphicx}
\usepackage{gnuplot-lua-tikz}
\usepackage{pgfplots}
\usepackage{tikz}
\usepackage{amsmath,amsfonts,amssymb}
\usepackage{bm}
\usepackage{siunitx}

\makeatletter
\def\fgcaption{\def\@captype{figure}\caption}
\makeatother
\newcommand{\setsections}[3]{
\setcounter{section}{#1}
\setcounter{subsection}{#2}
\setcounter{subsubsection}{#3}
}
\newcommand{\mfig}[3][width=15cm]{
\begin{center}
\includegraphics[#1]{#2}
\fgcaption{#3 \label{fig:#2}}
\end{center}
}
\newcommand{\gnu}[2]{
\begin{figure}[hptb]
\begin{center}
\input{#2}
\caption{#1}
\label{fig:#2}
\end{center}
\end{figure}
}

\begin{document}
\title{磁性物理学 レポート No.2}
\author{82311971 佐々木良輔}
\date{}
\maketitle
相互作用のない古典的な原子磁気モーメント$\bm m$を考える.
外部磁場$\bm H$中でのエネルギーは, $\bm H$と$\bm m$のなす角を$\theta$として
\begin{align}
  E=-{\bm m}\cdot{\bm H}=-mH\cos\theta
\end{align}
で与えられる.このとき温度$T$の$\bm m$が$\theta$を向く確率はボルツマン因子
\begin{align}
  e^{-\beta E}=e^{mH\cos\theta/k_BT}
\end{align}
に比例する.また角度$\theta$から$\theta+d\theta$の間の面積を$dA(\theta)$とすると,
全磁気モーメントのうち$\theta$から$\theta+d\theta$を向く割合は
\begin{align}
  P(\theta)d\theta=\frac{e^{-\beta E}dA(\theta)}{\int_0^\pi e^{-\beta E}dA(\theta)d\theta}
\end{align}
ここで$dA(\theta)$は図\ref{fig:dA.jpg}より
\begin{align}
  dA(\theta)=2\pi R^2\sin\theta d\theta
\end{align}
また$\bm m$は磁場方向に$\cos\theta$の大きさで寄与するので外部磁場方向の磁化$M$は,単位体積あたりの磁気モーメント数を$N$として
\begin{align}
  \begin{split}
    M&=Nm\langle\cos\theta\rangle\\
    &=Nm\int_0^\pi\cos\theta P(\theta)d\theta\\
    &=Nm\cfrac{\int_0^\pi e^{\beta mH\cos\theta}2\pi R^2\sin\theta\cos\theta d\theta}{\int_0^\pi e^{\beta mH\cos\theta}2\pi R^2\sin\theta d\theta}\\
    &=Nm\cfrac{\int_0^\pi e^{\beta mH\cos\theta}\sin\theta\cos\theta d\theta}{\int_0^\pi e^{\beta mH\cos\theta} \sin\theta d\theta}
  \end{split}
\end{align}
ここで分母は$\cos\theta=t$とすると$dt=-\sin\theta d\theta$より
\begin{align}
  \begin{split}
    \int_0^\pi e^{\beta mH\cos\theta} \sin\theta d\theta&=\int^1_{-1}e^{\beta mHt}dt\\
    &=\frac{e^{\beta mH}-e^{-\beta mH}}{\beta mH}\\
    &=\frac{2\sinh(\beta mH)}{\beta mH}
  \end{split}
\end{align}
また(6)式を両辺$\beta mH$で微分すると左辺は
\begin{align}
  \frac{d}{d(\beta mH)}\int_0^\pi e^{\beta mH\cos\theta} \sin\theta d\theta&=\int_0^\pi e^{\beta mH\cos\theta} \sin\theta\cos\theta d\theta
\end{align}
右辺は
\begin{align}
  \frac{d}{d(\beta mH)}\frac{2\sinh(\beta mH)}{\beta mH}&=2\frac{\beta mH\cosh(\beta mH)-\sinh(\beta mH)}{(\beta mH)^2}
\end{align}
したがって(5)式の分子は
\begin{align}
  \int_0^\pi e^{\beta mH\cos\theta} \sin\theta\cos\theta d\theta=2\frac{\cosh(\beta mH)}{\beta mH}-2\frac{\sinh(\beta mH)}{(\beta mH)^2}
\end{align}
以上から
\begin{align}
  \begin{split}
    M&=Nm\frac{2\frac{\cosh(\beta mH)}{\beta mH}-2\frac{\sinh(\beta mH)}{(\beta mH)^2}}{\frac{2\sinh(\beta mH)}{\beta mH}}\\
    &=Nm\left(\coth(\beta mH)-\frac{1}{\beta mH}\right)\\
    &=NmL(\beta mH)
  \end{split}
\end{align}
となる.
\mfig[width=6cm]{dA.jpg}{$\theta$から$\theta+d\theta$の面積}
\bibliography{ref.bib}
\end{document}