\documentclass[uplatex,a4j,11pt,dvipdfmx]{jsarticle}
\usepackage{listings,jvlisting}
\bibliographystyle{junsrt}

\usepackage{url}

\usepackage{graphicx}
\usepackage{gnuplot-lua-tikz}
\usepackage{pgfplots}
\usepackage{tikz}
\usepackage{amsmath,amsfonts,amssymb}
\usepackage{bm}
\usepackage{siunitx}

\makeatletter
\def\fgcaption{\def\@captype{figure}\caption}
\makeatother
\newcommand{\setsections}[3]{
\setcounter{section}{#1}
\setcounter{subsection}{#2}
\setcounter{subsubsection}{#3}
}
\newcommand{\mfig}[3][width=15cm]{
\begin{center}
\includegraphics[#1]{#2}
\fgcaption{#3 \label{fig:#2}}
\end{center}
}
\newcommand{\gnu}[2]{
\begin{figure}[hptb]
\begin{center}
\input{#2}
\caption{#1}
\label{fig:#2}
\end{center}
\end{figure}
}
\newcommand{\gaa}{\gamma_{AA}}
\newcommand{\gbb}{\gamma_{BB}}
\newcommand{\gab}{\gamma_{AB}}

\begin{document}
\title{磁性物理学 レポート No.4}
\author{82311971 佐々木良輔}
\date{}
\maketitle
\subsection*{(2)}
副格子$A$, $B$の磁化はそれぞれ
\begin{align}
  \begin{split}
    M_A&=\frac{C}{T}(H_{\rm ext}+H_w^A)\\
    &=\frac{C}{T}\left(H_{\rm ext}+{\gaa}M_A-{\gab}M_B\right)
  \end{split}
\end{align}
\begin{align}
  \begin{split}
    M_B&=\frac{C}{T}(H_{\rm ext}+H_w^B)\\
    &=\frac{C}{T}\left(H_{\rm ext}+{\gbb}M_B-{\gab}M_A\right)
  \end{split}
\end{align}
であった.ここで簡単のため$M_A=x$, $M_B=y$, $C{\gaa}/T=a$,
$C{\gbb}/T=b$, $C{\gab}/T=c$, $CH_{\rm ext}/T=h$と置くと
\begin{align}
    &\left\{
    \begin{array}{c}
      x=h+ax-cy\\
      y=h+by-cx
    \end{array}\right.\\
    \iff&\left\{
    \begin{array}{c}
      x=\cfrac{h-cy}{1-a}\\
      y=\cfrac{h-cx}{1-b}
    \end{array}\right.
\end{align}
(16)から
\begin{align}
  \begin{array}{cc}
    &y=\cfrac{h}{1-b}-\cfrac{c}{1-b}\cfrac{h-cy}{1-a}=\cfrac{h(1-a-c)+c^2y}{(1-a)(1-b)}\\
    \iff&y\left(1-\cfrac{c^2}{(1-a)(1-b)}\right)=\cfrac{h(1-a-c)}{(1-a)(1-b)}\\
    \iff&y=\cfrac{h(1-a-c)}{1-a-b+ab-c^2}
  \end{array}
\end{align}
$x$については$a$と$b$を入れ替えれば
\begin{align}
  x=\frac{h(1-b-c)}{1-a-b+ab-c^2}
\end{align}
である.正味の磁化は$M=M_A+M_B=x+y$なので
\begin{align}
  \begin{array}{cc}
    &x+y=\frac{h(2-a-b-2c)}{1-a-b+ab-c^2}\\
    \iff&\cfrac{h}{x+y}=\cfrac{1-a-b+ab-c^2}{2-a-b-2c}\\
    \iff&\cfrac{\frac{C}{T}H_{\rm ext}}{M}=\cfrac{1-\frac{C}{T}{\gaa}-\frac{C}{T}{\gbb}+\left(\frac{C}{T}\right)^2{\gaa}{\gbb}-\left(\frac{C}{T}\right)^2{\gab}^2}{2-\frac{C}{T}{\gaa}-\frac{C}{T}{\gbb}-2\frac{C}{T}{\gab}}\\
    \iff&\cfrac{H_{\rm ext}}{M}=\cfrac{\frac{T}{C}-({\gaa}+{\gbb})+\frac{C}{T}\left({\gaa}{\gbb}-{\gab}^2\right)}{2-\frac{C}{T}({\gaa}-{\gbb}-2{\gab})}
  \end{array}
\end{align}
を得る.
また$1/\chi_D=(2{\gab}-{\gaa}-{\gbb})/4$, $b=C({\gaa}-{\gbb})^2/8$, $\theta=C({\gaa}+{\gbb}+2{\gab})/2$とおいたとき
\begin{align}
  \begin{split}
    \frac{T}{2C}+\frac{1}{\chi_D}-\frac{b}{T-\theta}&=\frac{T}{2C}+\frac{2{\gab}-{\gaa}-{\gbb}}{4}-\frac{\frac{C}{4T}({\gaa}-{\gbb})^2}{2-\frac{C}{T}({\gaa}+{\gbb}+2{\gab})}\\
    &=\frac{1}{2-\frac{C}{T}(\gaa+{\gbb}+2{\gab})}\biggl(-\frac{C(\gaa-\gbb)^2}{4T}\\
      &\qquad+\frac{2\gab-\gaa-\gbb}{4}\left(2-\frac{C}{T}(\gaa+\gbb+2\gab)\right)\\
      &\qquad+\frac{T}{2C}\left(2-\frac{C}{T}(\gaa+\gbb+2\gab)\right)\biggr)\\
    &=\frac{1}{2-\frac{C}{T}(\gaa+{\gbb}+2{\gab})}
      \biggl(-\frac{C}{4T}\gaa^2+\frac{C}{2T}\gaa\gbb-\frac{C}{4T}\gbb^2\\
      &\qquad+\gab-\frac{\gaa}{2}-\frac{\gbb}{2}-\frac{C}{4T}\left(4\gab^2-(\gaa+\gbb)^2\right)\\
      &\qquad+\frac{T}{C}-\frac{1}{2}\left(\gaa+\gbb+2\gab\right)\biggr)\\
    &=\frac{1}{2-\frac{C}{T}(\gaa+{\gbb}+2{\gab})}
      \biggl(-\frac{C}{4T}\gaa^2+\frac{C}{2T}\gaa\gbb-\frac{C}{4T}\gbb^2\\
      &\qquad+\gab-\frac{\gaa}{2}-\frac{\gbb}{2}-\frac{C}{T}\gab^2+\frac{C}{4T}\left(\gaa^2+2\gaa\gbb+\gbb^2\right)\\
      &\qquad+\frac{T}{C}-\frac{1}{2}\left(\gaa+\gbb+2\gab\right)\biggr)\\
    &=\frac{1}{2-\frac{C}{T}(\gaa+{\gbb}+2{\gab})}
      \biggl(0+\frac{C}{T}\gaa\gbb-0\\
      &\qquad+0-\gaa-\gbb-\frac{C}{T}\gab^2+\frac{T}{C}\biggr)\\
    &=\frac{\frac{T}{C}-({\gaa}+{\gbb})+\frac{C}{T}\left({\gaa}{\gbb}-{\gab}^2\right)}{2-\frac{C}{T}({\gaa}-{\gbb}-2{\gab})}
  \end{split}
\end{align}
以上から
\begin{align}
  \frac{H_{\rm ext}}{M}=\frac{T}{2C}+\frac{1}{\chi_D}-\frac{b}{T-\theta}
\end{align}
である.
\end{document}