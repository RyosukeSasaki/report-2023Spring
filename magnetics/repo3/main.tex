\documentclass[uplatex,a4j,11pt,dvipdfmx]{jsarticle}
\usepackage{listings,jvlisting}
\bibliographystyle{junsrt}

\usepackage{url}

\usepackage{graphicx}
\usepackage{gnuplot-lua-tikz}
\usepackage{pgfplots}
\usepackage{tikz}
\usepackage{amsmath,amsfonts,amssymb}
\usepackage{bm}
\usepackage{siunitx}

\makeatletter
\def\fgcaption{\def\@captype{figure}\caption}
\makeatother
\newcommand{\setsections}[3]{
\setcounter{section}{#1}
\setcounter{subsection}{#2}
\setcounter{subsubsection}{#3}
}
\newcommand{\mfig}[3][width=15cm]{
\begin{center}
\includegraphics[#1]{#2}
\fgcaption{#3 \label{fig:#2}}
\end{center}
}
\newcommand{\gnu}[2]{
\begin{figure}[hptb]
\begin{center}
\input{#2}
\caption{#1}
\label{fig:#2}
\end{center}
\end{figure}
}
\newcommand{\gaa}{\gamma_{AA}}
\newcommand{\gbb}{\gamma_{BB}}
\newcommand{\gab}{\gamma_{AB}}

\begin{document}
\title{磁性物理学 レポート No.3}
\author{82311971 佐々木良輔}
\date{}
\maketitle
\subsection*{(1)}
図のように副格子Aの向く方向を正として,外部磁場は正の方向にかけるものとする.
外部磁場$H_{\rm ext}$が無い状態での副格子$A$, $B$磁化を$M_0$とする.
磁場を印加した際の副格子$A$, $B$それぞれについて,磁化の変化量を$\delta M_A$, $\delta M_B$とすると,
磁場中での副格子$A$, $B$の磁化は
\begin{align}
  \begin{split}
    M_A=M_0+\delta M_A\\
    M_B=-M_0+\delta M_B
  \end{split}
\end{align}
である.このとき副格子$A$が$B$から受ける分子場は
\begin{align}
  H_w^A=-\gamma M_B=-\gamma(-M_0+\delta M_B)
\end{align}
また$B$が$A$から受ける分子場は
\begin{align}
  H_w^B=-\gamma M_A=-\gamma(M_0+\delta M_A)
\end{align}
である.ここで図2のように外部磁場が無い状態での平衡位置を$\alpha_0$, $-\alpha_0$とする.
ただし外部磁場0のときの磁化が$M_0$なので$\alpha_0=\beta m\gamma M_0$である.
外部磁場を印加したときの平衡状態からのズレを$\delta\alpha_A$, $\delta\alpha_B$とすると
\begin{align}
  \begin{split}
    \delta\alpha_A&=\beta m(H_{\rm ext}+H_w^A)-\alpha_0\\
    &=\beta m(H_{\rm ext}-\gamma\delta M_B)+\beta m\gamma M_0-\alpha_0\\
    &=\beta m(H_{\rm ext}-\gamma\delta M_B)
  \end{split}
\end{align}
同様にして
\begin{align}
  \begin{split}
    \delta\alpha_B&=\beta m(H_{\rm ext}+H_w^B)-(-\alpha_0)\\
    &=\beta m(H_{\rm ext}-\gamma\delta M_A)
  \end{split}
\end{align}
である.ここで$\delta\alpha\ll1$のとき,
$M$を$\alpha_0$周りで1次までテイラー展開することで
\begin{align}
  M(\alpha_0+\delta\alpha)=NmL(\alpha_0)+NmL'(\alpha_0)\delta\alpha
\end{align}
これを用いて$\delta M_A$, $\delta M_B$は
\begin{align}
  \begin{split}
    \delta M_A&=M(\alpha_0+\delta\alpha_A)-M(\alpha_0)\\
    &=NmL'(\alpha_0)\beta m(H_{\rm ext}-\gamma\delta M_B)\\
    &=\frac{Nm^2}{k_BT}L'(\alpha_0)(H_{\rm ext}-\gamma\delta M_B)
  \end{split}
\end{align}
\begin{align}
  \begin{split}
    \delta M_B&=\frac{Nm^2}{k_BT}L'(\alpha_0)(H_{\rm ext}-\gamma\delta M_A)
  \end{split}
\end{align}
正味の磁化は$M=\delta M_A+\delta M_B$なので(7), (8)の両辺を足すと
\begin{align}
  \begin{array}{cc}
    &M=\cfrac{Nm^2}{k_BT}L'(\alpha_0)(2H_{\rm ext}-\gamma M)\\
    \iff&M\left(1+\cfrac{\gamma Nm^2L'(\alpha_0)}{k_BT}\right)=\cfrac{2Nm^2}{k_BT}L'(\alpha_0)H_{\rm ext}
  \end{array}
\end{align}
したがって磁気感受率は
\begin{align}
  \begin{split}
    \chi=\frac{M}{H_{\rm ext}}&=\frac{\frac{2Nm^2}{k_BT}L'(\alpha_0)}{1+\frac{\gamma Nm^2L'(\alpha_0)}{k_BT}}\\
    &=\frac{2Nm^2L'(\alpha_0)}{k_BT+\gamma Nm^2L'(\alpha_0)}
  \end{split}
\end{align}
となる.ここで$\alpha_0=0$では$L'(0)=1/3$なので
\begin{align}
  \begin{split}
    \chi&=\frac{\frac{2Nm^2}{3k_B}}{T+\frac{\gamma Nm^2}{3k_B}}=:\frac{C}{T+T_N}
  \end{split}
\end{align}
ここで$C=2Nm^2/3k_B$, $T_N=\gamma Nm^2/3k_B$とした.
$T=T_N$のときは
\begin{align}
  \chi=\frac{1}{\gamma}
\end{align}
となる.
\begin{center}
  \includegraphics[width=8cm]{1_coord.png}
  \fgcaption{反強磁性体に磁場を印加した際の磁化の挙動}
\end{center}
\begin{center}
  \includegraphics[width=12cm]{1_HM.png}
  \fgcaption{外部磁場による自己無撞着解の変化}
\end{center}
\end{document}