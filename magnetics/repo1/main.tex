\documentclass[uplatex,a4j,11pt,dvipdfmx]{jsarticle}
\usepackage{listings,jvlisting}
\bibliographystyle{junsrt}

\usepackage{url}

\usepackage{graphicx}
\usepackage{gnuplot-lua-tikz}
\usepackage{pgfplots}
\usepackage{tikz}
\usepackage{amsmath,amsfonts,amssymb}
\usepackage{bm}
\usepackage{siunitx}

\makeatletter
\def\fgcaption{\def\@captype{figure}\caption}
\makeatother
\newcommand{\setsections}[3]{
\setcounter{section}{#1}
\setcounter{subsection}{#2}
\setcounter{subsubsection}{#3}
}
\newcommand{\mfig}[3][width=15cm]{
\begin{center}
\includegraphics[#1]{#2}
\fgcaption{#3 \label{fig:#2}}
\end{center}
}
\newcommand{\gnu}[2]{
\begin{figure}[hptb]
\begin{center}
\input{#2}
\caption{#1}
\label{fig:#2}
\end{center}
\end{figure}
}

\begin{document}
\title{磁性物理学 レポート No.1}
\author{82311971 佐々木良輔}
\date{}
\maketitle
$(3d)^2$電子が取りうる量子数の組は
\begin{align}
  \begin{array}{llllll}
    (m_l,m_s)=&
    ^{(a)}(2,1/2),&
    ^{(b)}(1,1/2),&
    ^{(c)}(0,1/2),&
    ^{(d)}(-1,1/2),&
    ^{(e)}(-2,1/2),\\&
    ^{(f)}(2,-1/2),&
    ^{(g)}(1,-1/2),&
    ^{(h)}(0,-1/2),&
    ^{(i)}(-1,-1/2),&
    ^{(j)}(-2,-1/2)
  \end{array}
\end{align}
この内$M_L$, $M_S$が共に非負となる場合を列挙すると
\begin{align}
  \begin{array}{c|cc}
    &M_L&M_S\\
    \hline
    (a)+(b)&3&1\\
    (a)+(c)&2&1\\
    (a)+(d)&1&1\\
    (a)+(e)&0&1\\
    (a)+(f)&4&0\\
    (a)+(g)&3&0\\
    (a)+(h)&2&0\\
    (a)+(i)&1&0\\
    (a)+(j)&0&0\\
    (b)+(c)&1&1\\
    (b)+(d)&0&1\\
    (b)+(f)&3&0\\
    (b)+(g)&2&0\\
    (b)+(h)&1&0\\
    (b)+(i)&0&0\\
    (c)+(f)&2&0\\
    (c)+(g)&1&0\\
    (c)+(h)&0&0\\
    (d)+(f)&1&0\\
    (d)+(g)&0&0\\
    (e)+(f)&0&0\\
  \end{array}
\end{align}
となる.したがって$M_L$-$M_S$平面において各格子点の状態数は図\ref{fig:3d2_poly.png}(a)のようになり,これは図\ref{fig:3d2_poly.png}(b)から(f)のように分解される.
図\ref{fig:3d2_poly.png}(b)において$L=4,\ S=0$から$J=4$なので,多重項は$^1G_4$である.
図\ref{fig:3d2_poly.png}(c)において$L=3,\ S=1$から$J=4,\ 3,\ 2$なので,多重項は$^3F_4,\ ^3F_3,\ ^3F_2$である.
図\ref{fig:3d2_poly.png}(d)において$L=2,\ S=0$から$J=2$なので多重項は$^1D_2$である.
図\ref{fig:3d2_poly.png}(c)において$L=1,\ S=1$から$J=2,\ 1,\ 0$なので,多重項は$^3P_2,\ ^3P_1,\ ^3P_0$である.
図\ref{fig:3d2_poly.png}(b)において$L=0,\ S=0$から$J=0$なので多重項は$^1S_0$である.
また各多重項のLandeのg因子は
\begin{align}
  g=1+\frac{J(J+1)+S(S+1)-L(L+1)}{2J(J+1)}
\end{align}
より表\ref{tab:lande}のように計算される.
\mfig[width=14cm]{3d2_poly.png}{$M_L$-$M_S$平面における状態の分解,及びその多重項}
\begin{table}[h]
  \caption{各多重項のLandeのg因子}
  \label{tab:lande}
  \centering
  \begin{tabular}{c|ccc|c}
    \hline
    多重項&$L$&$S$&$J$&$g$\\
    \hline \hline
    $^1G_4$&4&0&4&1\\
    $^3F_4$&3&1&4&$5/4$\\
    $^3F_3$&3&1&3&$13/12$\\
    $^3F_2$&3&1&2&$2/3$\\
    $^1D_2$&2&0&2&1\\
    $^3P_2$&1&1&2&$3/2$\\
    $^3P_1$&1&1&1&$3/2$\\
    $^3P_0$&1&1&0&$3/2$\\
    $^1S_0$&0&0&0&$3/2$\\
    \hline
  \end{tabular}
\end{table}
\bibliography{ref.bib}
\end{document}