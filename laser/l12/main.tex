\documentclass[uplatex,a4j,11pt,dvipdfmx]{jsarticle}
\usepackage{listings,jvlisting}
\bibliographystyle{junsrt}

\usepackage{url}

\usepackage{graphicx}
\usepackage{gnuplot-lua-tikz}
\usepackage{pgfplots}
\usepackage{tikz}
\usepackage{amsmath,amsfonts,amssymb}
\usepackage{bm}
\usepackage{siunitx}

\makeatletter
\def\fgcaption{\def\@captype{figure}\caption}
\makeatother
\newcommand{\setsections}[3]{
\setcounter{section}{#1}
\setcounter{subsection}{#2}
\setcounter{subsubsection}{#3}
}
\newcommand{\mfig}[3][width=15cm]{
\begin{center}
\includegraphics[#1]{#2}
\fgcaption{#3 \label{fig:#2}}
\end{center}
}
\newcommand{\gnu}[2]{
\begin{figure}[hptb]
\begin{center}
\input{#2}
\caption{#1}
\label{fig:#2}
\end{center}
\end{figure}
}

\begin{document}
\title{レーザー物理学 レポート No.12}
\author{82311971 佐々木良輔}
\date{}
\maketitle
\section*{問20}
簡単のため以下では$\Omega_0^2=y$とする.また$B=C$とする.このとき与式は部分分数分解を行うと
\begin{align}
  \begin{split}
    dt&=\frac{1+By}{A(1-By)y}dy\\
    &=\frac{1}{A}\left(\frac{1}{y}+\frac{2B}{1-By}\right)dy
  \end{split}
\end{align}
となるので.両辺積分し
\begin{align}
  {\everymath{\displaystyle}
  \begin{array}{cc}
    &\int dt=\int\frac{1}{A}\left(\frac{1}{y}+\frac{2B}{1-By}\right)dy\\
    \iff&t=\frac{1}{A}\left(\log y-2\log(1-By)\right)+C
    \iff&t=\frac{1}{A}\left(\log \frac{y}{(1-By)^2}\right)+C
  \end{array}
  }
\end{align}
ただし$C$は積分定数である.
\section*{問21}
$f(\omega)$を以下で定義する.
\begin{align}
  f(\omega)=\sum_n\delta\left(\omega-\frac{2n\pi}{t_r}\right)
\end{align}
これは周期$2\pi/t_r$の周期関数であり,フーリエ級数展開を用いて
\begin{align}
  f(\omega)=\sum_n\tilde{f}_ne^{in\omega t_r}
\end{align}
と表される.
$\tilde{f}_n$はフーリエ係数であり
\begin{align}
  \begin{split}
    \tilde{f}_n&=\frac{t_r}{2\pi}\int_{-\pi/t_r}^{\pi/t_r}f(\omega)e^{-in\omega t_r}d\omega\\
    &=\frac{t_r}{2\pi}\int_{-\pi/t_r}^{\pi/t_r}\sum_k\delta\left(\omega-\frac{2k\pi}{t_r}\right)e^{-in\omega t_r}d\omega
  \end{split}
\end{align}
ここで積分区間$[-\pi/t_r,\pi/t_r]$には$k=0$の項しか含まれないため
\begin{align}
  \begin{split}
    \tilde{f}_n&=\frac{t_r}{2\pi}\int_{-\pi/t_r}^{\pi/t_r}\delta\left(\omega\right)e^{-in\omega t_r}d\omega\\
    &=\frac{t_r}{2\pi}
  \end{split}
\end{align}
したがって
\begin{align}
  \begin{array}{cc}
    &f(\omega)=\displaystyle\sum_n\cfrac{t_r}{2\pi}e^{in\omega t_r}=\displaystyle\sum_n\delta\left(\omega-\frac{2n\pi}{t_r}\right)\\
    \iff&\displaystyle\sum_ne^{in\omega t_r}=\cfrac{2\pi}{t_r}\displaystyle\sum_n\delta\left(\omega-\frac{2n\pi}{t_r}\right)
  \end{array}
\end{align}
が示される.
\end{document}