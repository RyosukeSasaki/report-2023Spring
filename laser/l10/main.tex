\documentclass[uplatex,a4j,11pt,dvipdfmx]{jsarticle}
\usepackage{listings,jvlisting}
\bibliographystyle{junsrt}

\usepackage{url}

\usepackage{graphicx}
\usepackage{gnuplot-lua-tikz}
\usepackage{pgfplots}
\usepackage{tikz}
\usepackage{amsmath,amsfonts,amssymb}
\usepackage{bm}
\usepackage{siunitx}

\makeatletter
\renewcommand{\theequation}{%
\thesection.\arabic{equation}}
\@addtoreset{equation}{section}
\makeatother

\makeatletter
\def\fgcaption{\def\@captype{figure}\caption}
\makeatother
\newcommand{\setsections}[3]{
\setcounter{section}{#1}
\setcounter{subsection}{#2}
\setcounter{subsubsection}{#3}
}
\newcommand{\mfig}[3][width=15cm]{
\begin{center}
\includegraphics[#1]{#2}
\fgcaption{#3 \label{fig:#2}}
\end{center}
}
\newcommand{\gnu}[2]{
\begin{figure}[hptb]
\begin{center}
\input{#2}
\caption{#1}
\label{fig:#2}
\end{center}
\end{figure}
}

\begin{document}
\title{レーザー物理学 レポート No.7}
\author{82311971 佐々木良輔}
\date{}
\maketitle
\section*{問17}
\setcounter{section}{17}
\setcounter{equation}{0}
閉じた2準位系でのレート方程式は, ポンピングのレートを$\Gamma$, 状態2から状態1への遷移レートを$\gamma_{21}$とすると
\begin{align}
  \begin{split}
    \frac{dN_1}{dt}&=-\Gamma N_1+(\Gamma+\gamma_{21})N_2\\
    \frac{dN_2}{dt}&=\Gamma N_1-(\Gamma+\gamma_{21})N_2
  \end{split}
\end{align}
ただし準位1から準位2への緩和は無視している.定常状態においてこれらの式は時間微分項を0とすることで
\begin{align}
  \Gamma N_1=(\Gamma+\gamma_{21})N_2
\end{align}
である.ここで反転分布が形成される条件は$N_2/N_1>1$なので,これを計算すると
\begin{align}
  \frac{N_2}{N_1}=\frac{\Gamma}{\Gamma+\gamma_{21}}
\end{align}
ここで$\Gamma$, $\gamma_{21}$は共に正なので$N_2/N_1\leq 1$が常に成り立つ.
したがって2準位系においては反転分布は作れない.
\section*{問18}
\setcounter{section}{18}
\setcounter{equation}{0}
熱平衡状態において$N_0$と$N_1$の比は
\begin{align}
  \frac{N_1}{N_0}=e^{-\frac{W_1-W_0}{k_BT}}
\end{align}
で与えられる.また$W_0$, $W_1$間での緩和のみを考えるとレート方程式は
\begin{align}
  \begin{split}
    \frac{dN_0}{dt}&=-\gamma_{01}+\gamma_{10}N_1\\
    \frac{dN_1}{dt}&=\gamma_{01}-\gamma_{10}N_1
  \end{split}
\end{align}
したがって定常状態において
\begin{align}
  \begin{array}{cc}
    &\gamma_{01}N_0=\gamma_{10}N_1\\
    \iff&\cfrac{\gamma_{01}}{\gamma_{10}}=\cfrac{N_1}{N_0}
  \end{array}
\end{align}
(18.1), (18.3)から
\begin{align}
  \frac{\gamma_{01}}{\gamma_{10}}=e^{-\frac{W_1-W_0}{k_BT}}
\end{align}
となる.
\end{document}