\documentclass[uplatex,a4j,11pt,dvipdfmx]{jsarticle}
\usepackage{listings,jvlisting}
\bibliographystyle{junsrt}

\usepackage{url}

\usepackage{graphicx}
\usepackage{gnuplot-lua-tikz}
\usepackage{pgfplots}
\usepackage{tikz}
\usepackage{amsmath,amsfonts,amssymb}
\usepackage{bm}
\usepackage{siunitx}

\makeatletter
\def\fgcaption{\def\@captype{figure}\caption}
\makeatother
\newcommand{\setsections}[3]{
\setcounter{section}{#1}
\setcounter{subsection}{#2}
\setcounter{subsubsection}{#3}
}
\newcommand{\mfig}[3][width=15cm]{
\begin{center}
\includegraphics[#1]{#2}
\fgcaption{#3 \label{fig:#2}}
\end{center}
}
\newcommand{\gnu}[2]{
\begin{figure}[hptb]
\begin{center}
\input{#2}
\caption{#1}
\label{fig:#2}
\end{center}
\end{figure}
}

\begin{document}
\title{レーザー物理学 レポート No.13}
\author{82311971 佐々木良輔}
\date{}
\maketitle
$n_1=n_2$, $E_1^*=E_1$のとき,微分方程式は
\begin{align}
  \frac{dE_1}{dz}&=iAE_1E_2e^{ik'z}\\
  \frac{dE_2}{dz}&=iAE_1^2e^{-ik'z}
\end{align}
ただし$k'=k_2-2k_1$, $A=\varepsilon_0\mu_0\chi_{xxx}^{(2)}\omega_1c/2n_1$とした.
ここで(1)式の両辺に$2E_1$を掛け,$E_1^2=f$, $E_2e^{ik'z}=g$とすると
\begin{align}
  \everymath{\displaystyle}
  \begin{array}{cc}
    &E_1\frac{dE_1}{dz}=2iAE_1^2E_2e^{ik'z}\\
    \iff&\frac{dE_1^2}{dz}=2iAE_1^2E_2e^{ik'z}\\
    \iff&\frac{df}{dz}=2iAfg
  \end{array}
\end{align}
また(2)式は
\begin{align}
  \everymath{\displaystyle}
  \begin{array}{cc}
    &\frac{dE_2}{dz}=iAE_1^2e^{-ik'z}\\
    \iff&\frac{dg}{dz}=iAf
  \end{array}
\end{align}
となる.ここで
\begin{align}
  f(z)=E_0^2(1-\tanh^2(AE_0z))\\
  g(z)=iE_0\tanh(AE_0z)
\end{align}
とすると
\begin{align}
  \begin{split}
    \frac{df}{dz}&=-2E_0^2\tanh(AE_0z)\frac{AE_0(e^{AE_0z}+e^{-AE_0z})^2-A(e^{AE_0z}-e^{-AE_0z})^2}{(e^{AE_0z}+e^{-AE_0z})^2}\\
    &=-2AE_0\tanh(AE_0z)E_0^2\left(1-\tanh^2(AE_0z)\right)\\
    &=2iA(iE_0\tanh(AE_0z))E_0^2\left(1-\tanh^2(AE_0z)\right)\\
    &=2iAfg
  \end{split}
\end{align}
\begin{align}
  \begin{split}
    \frac{dg}{dz}&=iE_0\frac{AE_0(e^{AE_0z}+e^{-AE_0z})^2-AE_0(e^{AE_0z}-e^{-AE_0z})^2}{(e^{AE_0z}+e^{-AE_0z})^2}\\
    &=iAf
  \end{split}
\end{align}
となり(3), (4)の解になっている.
また
\begin{align}
  f(0)&=E_1^2(0)=E_0^2(1-0)=E_0^2\\
  g(0)&=E_2(0)e^0=0
\end{align}
であり,境界条件を満たす.
したがって
\begin{align}
  E_2=iE_0\tanh(AE_0z)e^{-ik'z}
\end{align}
であり,強度は
\begin{align}
  |E_2|^2=E_0^2\tanh^2(AE_0z)
\end{align}
となる.
\end{document}