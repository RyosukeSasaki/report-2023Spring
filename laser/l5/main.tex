\documentclass[uplatex,a4j,11pt,dvipdfmx]{jsarticle}
\usepackage{listings,jvlisting}
\bibliographystyle{junsrt}

\usepackage{url}

\usepackage{graphicx}
\usepackage{gnuplot-lua-tikz}
\usepackage{pgfplots}
\usepackage{tikz}
\usepackage{amsmath,amsfonts,amssymb}
\usepackage{bm}
\usepackage{siunitx}

\makeatletter
\def\fgcaption{\def\@captype{figure}\caption}
\makeatother
\newcommand{\setsections}[3]{
\setcounter{section}{#1}
\setcounter{subsection}{#2}
\setcounter{subsubsection}{#3}
}
\newcommand{\mfig}[3][width=15cm]{
\begin{center}
\includegraphics[#1]{#2}
\fgcaption{#3 \label{fig:#2}}
\end{center}
}
\newcommand{\gnu}[2]{
\begin{figure}[hptb]
\begin{center}
\input{#2}
\caption{#1}
\label{fig:#2}
\end{center}
\end{figure}
}

\begin{document}
\title{レーザー物理学 レポート No.4}
\author{82311971 佐々木良輔}
\date{}
\maketitle
\section*{問6(1)}
図\ref{fig:laser_05_q61.pdf}のようにミラーは屈折率$n_2>n_1$のガラスと金属の反射部からなるため,
共振器内部での反射は固定端反射,共振器外部での反射は自由端反射となる.
したがって反射光の振幅は$2kL=\phi$をもちいて
\begin{align}
  \begin{split}
    E_r&=E_0e^{i(-kz-\omega t)}\left(r_1+t_1(-r_2)t_1e^{i\phi}
    +t_1(-r_2)(-r_1)(-r_2)t_1e^{2i\phi}+\cdots\right)\\
    &=E_0e^{i(-kz-\omega t)}\left(r_1-t_1^2r_2e^{i\phi}\sum_{n=0}^\infty\left(r_1r_2e^{i\phi}\right)^n\right)\\
    &=E_0e^{i(-kz-\omega t)}\left(r_1-\frac{t_1^2r_2r^{i\phi}}{1-r_1r_2e^{i\phi}}\right)
  \end{split}
\end{align}
となる.したがって
\begin{align}
  \begin{split}
    \frac{|E_r|^2}{|E_0|^2}=&
    \left(r_1-\frac{t_1^2r_2e^{i\phi}}{1-r_1r_2e^{i\phi}}\right)
    \left(r_1-\frac{t_1^2r_2e^{-i\phi}}{1-r_1r_2e^{-i\phi}}\right)\\
    =&r_1^2-r_1\left(
      \frac{t_1^2r_2e^{i\phi}}{1-r_1r_2e^{i\phi}}+
      \frac{t_1^2r_2e^{-i\phi}}{1-r_1r_2e^{-i\phi}}
    \right)
    +\frac{t_1^2r_2e^{i\phi}}{1-r_1r_2e^{i\phi}}
    \frac{t_1^2r_2e^{-i\phi}}{1-r_1r_2e^{-i\phi}}\\
    =&r_1^2-r_1\frac{t_1^2r_2e^{i\phi}\left(1-r_1r_2e^{-i\phi}\right)+
    t_1^2r_2e^{-i\phi}\left(1-r_1r_2e^{i\phi}\right)}{\left(1-r_1r_2e^{i\phi}\right)\left(1-r_1r_2e^{-i\phi}\right)}\\
    &+\frac{t_1^4r_2^2}{\left(1-r_1r_2e^{i\phi}\right)\left(1-r_1r_2e^{-i\phi}\right)}\\
    =&r_1^2-r_1r_2t_1^2\frac{2\cos\phi-2r_1r_2}{1-2r_1r_2\cos\phi+r_1^2r_2^2}+
    \frac{r_2^2t_1^4}{1-2r_1r_2\cos\phi+r_1^2r_2^2}\\
    =&\frac{r_1^2\left(1-2r_1r_2\cos\phi+r_1^2r_2^2\right)-2r_1r_2t_1^2\cos\phi+2r_1^2r_2^2t_1^2+r_2^2t_1^4}
    {1-2r_1r_2\cos\phi+r_1^2r_2^2}\\
    =&\frac{r_1^2-2r_1^3r_2\cos\phi-2r_1r_2t_1^2\cos\phi+r_2^2(r_1^2+t_1^2)^2}{1-2r_1r_2\cos\phi+r_1^2r_2^2}\\
    =&\frac{r_1^2+(r_1^2+t_1^2)\left(r_2^2(r_1^2+t_1^2)-2r_1r_2\cos\phi\right)}
    {1-2r_1r_2\cos\phi+r_1^2r_2^2}\\
    =&\frac{R_1+(R_1+T_1)\left(R_2(R_1+T_1)-2\sqrt{R_1R_2}\cos\phi\right)}{1-2\sqrt{R_1R_2}\cos\phi+R_1R_2}
  \end{split}
\end{align}
したがって反射光強度は
\begin{align}
  \begin{split}
    I_r&=\frac{\varepsilon_0c}{2}|E_r|^2\\
    &=\frac{\varepsilon_0c}{2}|E_0^2|\frac{R_1+(R_1+T_1)\left(R_2(R_1+T_1)-2\sqrt{R_1R_2}\cos\phi\right)}{1-2\sqrt{R_1R_2}\cos\phi+R_1R_2}
  \end{split}
\end{align}
となる.ここで$R+T=1$とすると
\begin{align}
  \begin{split}
    \frac{|E_r|^2}{|E_0|^2}=&
    \frac{R_1+R_2-2\sqrt{R_1R_2}\cos\phi}{1-2\sqrt{R_1R_2}\cos\phi+R_1R_2}
  \end{split}
\end{align}
よって
\begin{align}
  \begin{split}
    \frac{|E_r|^2}{|E_0|^2}+\frac{|E_t|^2}{|E_0|^2}=&
    \frac{R_1+R_2-2\sqrt{R_1R_2}\cos\phi}{1-2\sqrt{R_1R_2}\cos\phi+R_1R_2}+
    \frac{(1-R_1)(1-R_2)}{1-2\sqrt{R_1R_2}\cos\phi+R_1R_2}\\
    =&1
  \end{split}
\end{align}
したがって反射光強度と透過光強度の合計は
\begin{align}
  I_t+I_r=\frac{\varepsilon_0c}{2}|E_0|^2
\end{align}
となり,入射光強度と等しい.
\mfig[width=6cm]{laser_05_q61.pdf}{共振器での反射の模式図}
\section*{問6(2)}
$R_1=R_2=R$のとき
\begin{align}
  \begin{split}
    I=&\frac{I_{\rm in}T_1}{1-2R\cos\phi+R^2}\\
    =&\frac{I_{\rm in}T_1}{1+R^2-2R\left(1-2\sin^2\frac{\phi}{2}\right)}\\
    =&\frac{I_{\rm in}T_1}{(1-R)^2+4R\sin^2\frac{\phi}{2}}
  \end{split}
\end{align}
より,強度の最大値は$I_{\rm in}T_1/(1-R)^2$である.
したがって強度が半分になるとき
\begin{eqnarray}
  &\frac{I_{\rm in}T_1}{2(1-R)^2}=\frac{I_{\rm in}T_1}{(1-R)^2+4R\sin^2\frac{\phi}{2}}\nonumber\\
  \iff&4R\sin^2\frac{\phi}{2}=(1-R)^2\nonumber\\
  \iff&\sin\frac{\phi}{2}=\frac{1-R}{2\sqrt{R}}\nonumber\\
\end{eqnarray}
ここで$\phi/2\ll 1$として展開すると
\begin{align}
  \begin{split}
    \frac{\phi}{2}&=\frac{1-R}{2\sqrt{R}}
  \end{split}
\end{align}
ここで$\phi=2kL$, $k=\omega/c$より
\begin{eqnarray}
  &L\frac{\Delta\omega}{c}=\frac{1-R}{2\sqrt{R}}\nonumber\\
  \iff&\Delta\omega=\frac{(1-R)c}{2L\sqrt{R}}
\end{eqnarray}
となる.次に${\rm FSR}=c/2L$より
\begin{align}
  F=\frac{\rm FSR}{2\Delta\nu}=\frac{\pi}{\Delta\omega}\frac{c}{2L}=\frac{\pi\sqrt{R}}{1-R}
\end{align}
である.また$R\simeq 1$より
\begin{align}
  Q=\frac{\omega_0}{2\Delta\omega}=\frac{\omega_0}{2}\frac{2L\sqrt{R}}{(1-R)c}=\frac{\omega_0L\sqrt{R}}{(1-R)c}\simeq\frac{\omega_0L}{(1-R)c}
\end{align}
である.一方でQ値の定義は
\begin{align}
  Q=2\pi\frac{共振器内のエネルギー}{1周期あたりのエネルギー損失}
\end{align}
であった.ここで共振器内のエネルギーは共振器の断面籍を$S$として
\begin{align}
  \frac{1}{2}\varepsilon_0|E|^2\times LS=\frac{1}{2}\varepsilon_0ILS=\frac{1}{2}\varepsilon_0\frac{TI_{\rm in}}{(1-R)^2}LS=\frac{1}{2}\varepsilon_0\frac{I_{\rm in}LS}{1-R}
\end{align}
また前問の結果から共振器から失われる光の強度は入射光強度$I_{\rm in}$に等しく,また$T=2\pi/\omega_0$より1周期あたりに共振器から失われるエネルギーは
\begin{align}
  \int^{2\pi/\omega_0}_0dt\frac{1}{2}c\varepsilon_0I_{\rm in}\times S=c\varepsilon_0\frac{\pi I_{\rm in}}{\omega_0}S
\end{align}
以上から
\begin{align}
  Q=2\pi\frac{\frac{1}{2}\varepsilon_0\frac{I_{\rm in}LS}{1-R}}{c\varepsilon_0\frac{\pi I_{\rm in}}{\omega_0}S}=\frac{\omega_0 L}{c(1-R)}
\end{align}
となる.
\section*{問7(1)}
ミラー1での強度反射率,強度透過率が$R$, $T$, 光が往復する間の強度損失が$\kappa$のとき,
電場振幅は図\ref{fig:laser_05_q71.pdf}のようになる.
これは前問において$r_1=\sqrt{R}$, $r_2=\sqrt{\kappa}$とした場合に等しい.
したがって共振器内の光強度は,共鳴状態において
\begin{align}
  I=\frac{(1-R)I_{\rm in}}{(1-\sqrt{\kappa R})^2}
\end{align}
となる.ここで
\begin{align}
  \frac{dI}{dR}=\frac{\kappa-\sqrt{\kappa R}}{\sqrt{\kappa R}(1-\sqrt{\kappa R})^3}I_{\rm in}
\end{align}
である. $0\leq R\leq 1$, $0\leq \kappa\leq 1$より$0\leq\sqrt{\kappa R}\leq 1$なので
$\sqrt{\kappa R}(1-\sqrt{\kappa R})^3>0$である.
したがって$R\leq\kappa$のとき$dI/dR\geq0$, $\kappa<R$のとき$dI/dR<0$である.
したがって$R=\kappa$において$I$は最大値を取る.
\mfig[width=8cm]{laser_05_q71.pdf}{損失$\kappa$のときの電場}
\section*{問7(2)}
\end{document}