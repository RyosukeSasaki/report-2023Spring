\documentclass[uplatex,a4j,11pt,dvipdfmx]{jsarticle}
\usepackage{listings,jvlisting}
\bibliographystyle{junsrt}

\usepackage{url}

\usepackage{graphicx}
\usepackage{gnuplot-lua-tikz}
\usepackage{pgfplots}
\usepackage{tikz}
\usepackage{amsmath,amsfonts,amssymb}
\usepackage{bm}
\usepackage{siunitx}

\makeatletter
\def\fgcaption{\def\@captype{figure}\caption}
\makeatother
\newcommand{\setsections}[3]{
\setcounter{section}{#1}
\setcounter{subsection}{#2}
\setcounter{subsubsection}{#3}
}
\newcommand{\mfig}[3][width=15cm]{
\begin{center}
\includegraphics[#1]{#2}
\fgcaption{#3 \label{fig:#2}}
\end{center}
}
\newcommand{\gnu}[2]{
\begin{figure}[hptb]
\begin{center}
\input{#2}
\caption{#1}
\label{fig:#2}
\end{center}
\end{figure}
}
\numberwithin{equation}{section}

\begin{document}
\title{レーザー物理学 レポートNo.2}
\author{82311971 佐々木良輔}
\date{}
\maketitle
\setcounter{section}{3}
\section*{問3}
$E_y,\ B_x,\ B_y,\ B_z$はぞれぞれ${\rm e}^{-i\omega t}$という項を持つとする.すなわち
\begin{align}
  \frac{\partial E_y}{\partial t}=-i\omega E_y
\end{align}
であり$B_x,\ B_y,\ B_z$についても同様である.ここでMaxwell方程式${\rm rot}\vec{E}=-\partial \vec{B}/\partial t$から
\begin{align}
  \label{rotE}
  \left\{
    \begin{array}[1]{r}
        -\cfrac{\partial E_y}{\partial z}=i\omega B_x\\
        \cfrac{\partial E_x}{\partial z}=i\omega B_y\\
        \cfrac{\partial E_y}{\partial x}-\cfrac{\partial E_x}{\partial y}=i\omega B_z
    \end{array}
  \right.
\end{align}
また${\rm rot}\vec{B}=\varepsilon_0\mu_0\partial \vec{E}/\partial t$から
\begin{align}
  \label{rotB}
  \left\{
    \begin{array}[1]{l}
      \cfrac{\partial B_z}{\partial y}-\cfrac{\partial B_y}{\partial z}=-i\omega\varepsilon_0\mu_0E_x\\
      \cfrac{\partial B_x}{\partial z}-\cfrac{\partial B_z}{\partial x}=-i\omega\varepsilon_0\mu_0E_y\\
      \cfrac{\partial B_y}{\partial x}-\cfrac{\partial B_x}{\partial y}=0
    \end{array}
  \right.
\end{align}
である.ただし$E_z=0$を用いた.
まず(\ref{rotE}) 第2式から
\begin{align}
  \label{By}
  \begin{split}
    B_y
    &=\frac{1}{i\omega}\frac{\partial E_x}{\partial z}\\
    &=\frac{1}{i\omega}\left(\frac{\partial u}{\partial z}{\rm e}^{i(kz-\omega t)}+iku{\rm e}^{i(kz-\omega t)}\right)\\
    &=\frac{{\rm e}^{i(kz-\omega t)}}{i\omega}\left(\frac{\partial u}{\partial z}+iku\right)
  \end{split}
\end{align}
である.
次に(\ref{rotB}) 第3式に(\ref{By})を代入し
\begin{align}
  \begin{split}
    \frac{\partial B_x}{\partial y}&=\frac{\partial B_y}{\partial x}\\
    &=\frac{1}{i\omega}\left(\frac{\partial^2u}{\partial x\partial z}+ik\frac{\partial u}{\partial x}\right){\rm e}^{i(kz-\omega t)}
  \end{split}
\end{align}
両辺を$y$で積分し
\begin{align}
  \label{Bx}
  B_x=\frac{{\rm e}^{i(kz-\omega t)}}{i\omega}\int dy\left(\frac{\partial^2u}{\partial x\partial z}+ik\frac{\partial u}{\partial x}\right)
\end{align}
である.
次に(\ref{rotE}) 第1式に(\ref{Bx})を代入し
\begin{align}
    \frac{\partial E_y}{\partial z}=-{\rm e}^{i(kz-\omega t)}\int dy\left(\frac{\partial^2u}{\partial x\partial z}+ik\frac{\partial u}{\partial x}\right)
\end{align}
両辺を$z$で積分し
\begin{align}
  \label{Ey}
  E_y=-\int dz\ {\rm e}^{i(kz-\omega t)}\int dy\left(\frac{\partial^2u}{\partial x\partial z}+ik\frac{\partial u}{\partial x}\right)
\end{align}
である.
次に(\ref{rotE}) 第3式に(\ref{Ey})を代入し
\begin{align}
  \begin{split}
    i\omega B_z&=-\frac{\partial}{\partial x}\left(\int dz\ {\rm e}^{i(kz-\omega t)}\int dy\left(\frac{\partial^2u}{\partial x\partial z}+ik\frac{\partial u}{\partial x}\right)\right)
    -\frac{\partial u}{\partial y}{\rm e}^{i(kz-\omega t)}\\
  \end{split}
\end{align}
以下では積分と偏微分が交換可能であるとする.
\begin{align}
  \begin{split}
    B_z&=-\frac{1}{i\omega}\left(\int dz\ {\rm e}^{i(kz-\omega t)}\int dy\left(
      \frac{\partial^3u}{\partial x^2\partial z}+ik\frac{\partial^2u}{\partial x^2}
    \right)+\frac{\partial u}{\partial y}{\rm e}^{i(kz-\omega t)}\right)
  \end{split}
\end{align}
となる.以上から
\begin{align}
  \vec{E}=\left(
    \begin{array}[1]{c}
      \displaystyle
      u(x,y,z){\rm e}^{i(kz-\omega t)}\\
      \displaystyle
      -\int dz\ {\rm e}^{i(kz-\omega t)}\int dy\left(\frac{\partial^2u}{\partial x\partial z}+ik\frac{\partial u}{\partial x}\right)\\
      0
    \end{array}
  \right)
\end{align}
\begin{align}
  \vec{B}=\left(
    \begin{array}[1]{c}
      \displaystyle
      \frac{{\rm e}^{i(kz-\omega t)}}{i\omega}\int dy\left(\frac{\partial^2u}{\partial x\partial z}+ik\frac{\partial u}{\partial x}\right)\\
      \displaystyle
      \frac{{\rm e}^{i(kz-\omega t)}}{i\omega}\left(\frac{\partial u}{\partial z}+iku\right)\\
      \displaystyle
      -\frac{1}{i\omega}\left(\int dz\ {\rm e}^{i(kz-\omega t)}\int dy\left(
      \frac{\partial^3u}{\partial x^2\partial z}+ik\frac{\partial^2u}{\partial x^2}
      \right)+\frac{\partial u}{\partial y}{\rm e}^{i(kz-\omega t)}\right)\\
    \end{array}
  \right)
\end{align}
を得る.ここで電荷$\rho=0$であることから${\rm div}\vec{E}=0$となることを確認する.
\begin{align}
  \frac{\partial E_x}{\partial x}=\frac{\partial u}{\partial x}{\rm e}^{i(kz-\omega t)}
\end{align}
\begin{align}
  \label{dEy/dy}
  \begin{split}
    \frac{\partial E_y}{\partial y}&=-\frac{\partial}{\partial y}\int dz\ {\rm e}^{i(kz-\omega t)}\int dy\left(\frac{\partial^2u}{\partial x\partial z}+ik\frac{\partial u}{\partial x}\right)\\
    &=-\int dz\ {\rm e}^{i(kz-\omega t)}\frac{\partial}{\partial y}\int dy\left(\frac{\partial^2u}{\partial x\partial z}+ik\frac{\partial u}{\partial x}\right)\\
    &=-\int dz\ {\rm e}^{i(kz-\omega t)}\frac{\partial^2u}{\partial x\partial z}-\int dz\ {\rm e}^{i(kz-\omega t)}ik\frac{\partial u}{\partial x}\\
  \end{split}
\end{align}
ここで最右辺 第1項について部分積分を実行すると
\begin{align}
  \begin{split}
    \int dz\ {\rm e}^{i(kz-\omega t)}\frac{\partial^2u}{\partial x\partial z}={\rm e}^{i(kz-\omega t)}\frac{\partial u}{\partial x}-\int dz\ ik{\rm e}^{i(kz-\omega t)}\frac{\partial u}{\partial x}
  \end{split}
\end{align}
したがって(\ref{dEy/dy})は
\begin{align}
  \begin{split}
    \frac{\partial E_y}{\partial y}&=-\frac{\partial u}{\partial x}{\rm e}^{i(kz-\omega t)}+\int dz\ ik{\rm e}^{i(kz-\omega t)}\frac{\partial u}{\partial x}-\int dz\ {\rm e}^{i(kz-\omega t)}ik\frac{\partial u}{\partial x}\\
    &=-\frac{\partial u}{\partial x}{\rm e}^{i(kz-\omega t)}
  \end{split}
\end{align}
\begin{align}
  \frac{\partial E_z}{\partial z}=0
\end{align}
以上から
\begin{align}
  {\rm div}\vec{E}=\frac{\partial E_x}{\partial x}+\frac{\partial E_y}{\partial y}+\frac{\partial E_z}{\partial z}=0
\end{align}
を得る.また${\rm div}\vec{B}=0$についても同様に確認する.
\begin{align}
  \begin{split}
    \frac{\partial B_x}{\partial x}&=\frac{\partial}{\partial x}\frac{{\rm e}^{i(kz-\omega t)}}{i\omega}\int dy\left(\frac{\partial^2u}{\partial x\partial z}+ik\frac{\partial u}{\partial x}\right)\\
    &=\frac{{\rm e}^{i(kz-\omega t)}}{i\omega}\int dy\left(\frac{\partial^3u}{\partial x^2\partial z}+ik\frac{\partial^2 u}{\partial x^2}\right)
  \end{split}
\end{align}
\begin{align}
  \begin{split}
    \frac{\partial B_y}{\partial y}&=\frac{{\rm e}^{i(kz-\omega t)}}{i\omega}\left(\frac{\partial^2 u}{\partial z\partial y}+ik\frac{\partial u}{\partial y}\right)
  \end{split}
\end{align}
\begin{align}
  \begin{split}
    \frac{\partial B_z}{\partial z}&=-\frac{1}{i\omega}\frac{\partial}{\partial z}\left(\int dz\ {\rm e}^{i(kz-\omega t)}\int dy\left(
    \frac{\partial^3u}{\partial x^2\partial z}+ik\frac{\partial^2u}{\partial x^2}
    \right)+\frac{\partial u}{\partial y}{\rm e}^{i(kz-\omega t)}\right)\\
    &=-\frac{{\rm e}^{i(kz-\omega t)}}{i\omega}\int dy\left(
    \frac{\partial^3u}{\partial x^2\partial z}+ik\frac{\partial^2u}{\partial x^2}
    \right)
    -\frac{1}{i\omega}\left(
    \frac{\partial^2 u}{\partial y\partial z}{\rm e}^{i(kz-\omega t)}
    +ik\frac{\partial u}{\partial y}{\rm e}^{i(kz-\omega t)}
    \right)\\
    &=-\frac{{\rm e}^{i(kz-\omega t)}}{i\omega}\left(
    \int dy\left(
    \frac{\partial^3u}{\partial x^2\partial z}+ik\frac{\partial^2u}{\partial x^2}
    \right)
    +\frac{\partial^2 u}{\partial y\partial z}+ik\frac{\partial u}{\partial y}
    \right)
  \end{split}
\end{align}
以上から
\begin{align}
  {\rm div}\vec{B}=\frac{\partial B_x}{\partial x}+\frac{\partial B_y}{\partial y}+\frac{\partial B_z}{\partial z}=0
\end{align}
を得る.
\setcounter{section}{4}
\section*{問4}
\end{document}