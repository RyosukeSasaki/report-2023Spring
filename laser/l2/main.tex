\documentclass[uplatex,a4j,11pt,dvipdfmx]{jsarticle}
\usepackage{listings,jvlisting}
\bibliographystyle{junsrt}

\usepackage{url}

\usepackage{graphicx}
\usepackage{gnuplot-lua-tikz}
\usepackage{pgfplots}
\usepackage{tikz}
\usepackage{amsmath,amsfonts,amssymb}
\usepackage{bm}
\usepackage{siunitx}

\makeatletter
\def\fgcaption{\def\@captype{figure}\caption}
\makeatother
\newcommand{\setsections}[3]{
\setcounter{section}{#1}
\setcounter{subsection}{#2}
\setcounter{subsubsection}{#3}
}
\newcommand{\mfig}[3][width=15cm]{
\begin{center}
\includegraphics[#1]{#2}
\fgcaption{#3 \label{fig:#2}}
\end{center}
}
\newcommand{\gnu}[2]{
\begin{figure}[hptb]
\begin{center}
\input{#2}
\caption{#1}
\label{fig:#2}
\end{center}
\end{figure}
}
\numberwithin{equation}{section}

\begin{document}
\title{レーザー物理学 レポートNo.2}
\author{82311971 佐々木良輔}
\date{}
\maketitle
\setcounter{section}{3}
\section*{問3}
$E_y,\ B_x,\ B_y,\ B_z$はぞれぞれ${\rm e}^{-i\omega t}$という項を持つとする.すなわち
\begin{align}
  \frac{\partial E_y}{\partial t}=-i\omega E_y
\end{align}
であり$B_x,\ B_y,\ B_z$についても同様である.ここでMaxwell方程式${\rm rot}\vec{E}=-\partial \vec{B}/\partial t$から
\begin{align}
  \label{rotE}
  \left\{
    \begin{array}[1]{r}
        -\cfrac{\partial E_y}{\partial z}=i\omega B_x\\
        \cfrac{\partial E_x}{\partial z}=i\omega B_y\\
        \cfrac{\partial E_y}{\partial x}-\cfrac{\partial E_x}{\partial y}=i\omega B_z
    \end{array}
  \right.
\end{align}
また${\rm rot}\vec{B}=\varepsilon_0\mu_0\partial \vec{E}/\partial t$から
\begin{align}
  \label{rotB}
  \left\{
    \begin{array}[1]{l}
      \cfrac{\partial B_z}{\partial y}-\cfrac{\partial B_y}{\partial z}=-i\omega\varepsilon_0\mu_0E_x\\
      \cfrac{\partial B_x}{\partial z}-\cfrac{\partial B_z}{\partial x}=-i\omega\varepsilon_0\mu_0E_y\\
      \cfrac{\partial B_y}{\partial x}-\cfrac{\partial B_x}{\partial y}=0
    \end{array}
  \right.
\end{align}
である.ただし$E_z=0$を用いた.
まず(\ref{rotE}) 第2式から
\begin{align}
  \label{By}
  \begin{split}
    B_y
    &=\frac{1}{i\omega}\frac{\partial E_x}{\partial z}\\
    &=\frac{1}{i\omega}\left(\frac{\partial u}{\partial z}{\rm e}^{i(kz-\omega t)}+iku{\rm e}^{i(kz-\omega t)}\right)\\
    &=\frac{{\rm e}^{i(kz-\omega t)}}{i\omega}\left(\frac{\partial u}{\partial z}+iku\right)
  \end{split}
\end{align}
である.
次に(\ref{rotB}) 第3式に(\ref{By})を代入し
\begin{align}
  \begin{split}
    \frac{\partial B_x}{\partial y}&=\frac{\partial B_y}{\partial x}\\
    &=\frac{1}{i\omega}\left(\frac{\partial^2u}{\partial x\partial z}+ik\frac{\partial u}{\partial x}\right){\rm e}^{i(kz-\omega t)}
  \end{split}
\end{align}
両辺を$y$で積分し
\begin{align}
  \label{Bx}
  B_x=\frac{{\rm e}^{i(kz-\omega t)}}{i\omega}\int dy\left(\frac{\partial^2u}{\partial x\partial z}+ik\frac{\partial u}{\partial x}\right)
\end{align}
である.
次に(\ref{rotE}) 第1式に(\ref{Bx})を代入し
\begin{align}
    \frac{\partial E_y}{\partial z}=-{\rm e}^{i(kz-\omega t)}\int dy\left(\frac{\partial^2u}{\partial x\partial z}+ik\frac{\partial u}{\partial x}\right)
\end{align}
両辺を$z$で積分し
\begin{align}
  \label{Ey}
  E_y=-\int dz\ {\rm e}^{i(kz-\omega t)}\int dy\left(\frac{\partial^2u}{\partial x\partial z}+ik\frac{\partial u}{\partial x}\right)
\end{align}
である.
次に(\ref{rotE}) 第3式に(\ref{Ey})を代入し
\begin{align}
  \begin{split}
    i\omega B_z&=-\frac{\partial}{\partial x}\left(\int dz\ {\rm e}^{i(kz-\omega t)}\int dy\left(\frac{\partial^2u}{\partial x\partial z}+ik\frac{\partial u}{\partial x}\right)\right)
    -\frac{\partial u}{\partial y}{\rm e}^{i(kz-\omega t)}\\
  \end{split}
\end{align}
以下では積分と偏微分が交換可能であるとする.
\begin{align}
  \begin{split}
    B_z&=-\frac{1}{i\omega}\left(\int dz\ {\rm e}^{i(kz-\omega t)}\int dy\left(
      \frac{\partial^3u}{\partial x^2\partial z}+ik\frac{\partial^2u}{\partial x^2}
    \right)+\frac{\partial u}{\partial y}{\rm e}^{i(kz-\omega t)}\right)
  \end{split}
\end{align}
となる.以上から
\begin{align}
  \vec{E}=\left(
    \begin{array}[1]{c}
      \displaystyle
      u(x,y,z){\rm e}^{i(kz-\omega t)}\\
      \displaystyle
      -\int dz\ {\rm e}^{i(kz-\omega t)}\int dy\left(\frac{\partial^2u}{\partial x\partial z}+ik\frac{\partial u}{\partial x}\right)\\
      0
    \end{array}
  \right)
\end{align}
\begin{align}
  \vec{B}=\left(
    \begin{array}[1]{c}
      \displaystyle
      \frac{{\rm e}^{i(kz-\omega t)}}{i\omega}\int dy\left(\frac{\partial^2u}{\partial x\partial z}+ik\frac{\partial u}{\partial x}\right)\\
      \displaystyle
      \frac{{\rm e}^{i(kz-\omega t)}}{i\omega}\left(\frac{\partial u}{\partial z}+iku\right)\\
      \displaystyle
      -\frac{1}{i\omega}\left(\int dz\ {\rm e}^{i(kz-\omega t)}\int dy\left(
      \frac{\partial^3u}{\partial x^2\partial z}+ik\frac{\partial^2u}{\partial x^2}
      \right)+\frac{\partial u}{\partial y}{\rm e}^{i(kz-\omega t)}\right)\\
    \end{array}
  \right)
\end{align}
を得る.ここで電荷$\rho=0$であることから${\rm div}\vec{E}=0$となることを確認する.
\begin{align}
  \frac{\partial E_x}{\partial x}=\frac{\partial u}{\partial x}{\rm e}^{i(kz-\omega t)}
\end{align}
\begin{align}
  \label{dEy/dy}
  \begin{split}
    \frac{\partial E_y}{\partial y}&=-\frac{\partial}{\partial y}\int dz\ {\rm e}^{i(kz-\omega t)}\int dy\left(\frac{\partial^2u}{\partial x\partial z}+ik\frac{\partial u}{\partial x}\right)\\
    &=-\int dz\ {\rm e}^{i(kz-\omega t)}\frac{\partial}{\partial y}\int dy\left(\frac{\partial^2u}{\partial x\partial z}+ik\frac{\partial u}{\partial x}\right)\\
    &=-\int dz\ {\rm e}^{i(kz-\omega t)}\frac{\partial^2u}{\partial x\partial z}-\int dz\ {\rm e}^{i(kz-\omega t)}ik\frac{\partial u}{\partial x}\\
  \end{split}
\end{align}
ここで最右辺 第1項について部分積分を実行すると
\begin{align}
  \begin{split}
    \int dz\ {\rm e}^{i(kz-\omega t)}\frac{\partial^2u}{\partial x\partial z}={\rm e}^{i(kz-\omega t)}\frac{\partial u}{\partial x}-\int dz\ ik{\rm e}^{i(kz-\omega t)}\frac{\partial u}{\partial x}
  \end{split}
\end{align}
したがって(\ref{dEy/dy})は
\begin{align}
  \begin{split}
    \frac{\partial E_y}{\partial y}&=-\frac{\partial u}{\partial x}{\rm e}^{i(kz-\omega t)}+\int dz\ ik{\rm e}^{i(kz-\omega t)}\frac{\partial u}{\partial x}-\int dz\ {\rm e}^{i(kz-\omega t)}ik\frac{\partial u}{\partial x}\\
    &=-\frac{\partial u}{\partial x}{\rm e}^{i(kz-\omega t)}
  \end{split}
\end{align}
\begin{align}
  \frac{\partial E_z}{\partial z}=0
\end{align}
以上から
\begin{align}
  {\rm div}\vec{E}=\frac{\partial E_x}{\partial x}+\frac{\partial E_y}{\partial y}+\frac{\partial E_z}{\partial z}=0
\end{align}
を得る.また${\rm div}\vec{B}=0$についても同様に確認する.
\begin{align}
  \begin{split}
    \frac{\partial B_x}{\partial x}&=\frac{\partial}{\partial x}\frac{{\rm e}^{i(kz-\omega t)}}{i\omega}\int dy\left(\frac{\partial^2u}{\partial x\partial z}+ik\frac{\partial u}{\partial x}\right)\\
    &=\frac{{\rm e}^{i(kz-\omega t)}}{i\omega}\int dy\left(\frac{\partial^3u}{\partial x^2\partial z}+ik\frac{\partial^2 u}{\partial x^2}\right)
  \end{split}
\end{align}
\begin{align}
  \begin{split}
    \frac{\partial B_y}{\partial y}&=\frac{{\rm e}^{i(kz-\omega t)}}{i\omega}\left(\frac{\partial^2 u}{\partial z\partial y}+ik\frac{\partial u}{\partial y}\right)
  \end{split}
\end{align}
\begin{align}
  \begin{split}
    \frac{\partial B_z}{\partial z}&=-\frac{1}{i\omega}\frac{\partial}{\partial z}\left(\int dz\ {\rm e}^{i(kz-\omega t)}\int dy\left(
    \frac{\partial^3u}{\partial x^2\partial z}+ik\frac{\partial^2u}{\partial x^2}
    \right)+\frac{\partial u}{\partial y}{\rm e}^{i(kz-\omega t)}\right)\\
    &=-\frac{{\rm e}^{i(kz-\omega t)}}{i\omega}\int dy\left(
    \frac{\partial^3u}{\partial x^2\partial z}+ik\frac{\partial^2u}{\partial x^2}
    \right)
    -\frac{1}{i\omega}\left(
    \frac{\partial^2 u}{\partial y\partial z}{\rm e}^{i(kz-\omega t)}
    +ik\frac{\partial u}{\partial y}{\rm e}^{i(kz-\omega t)}
    \right)\\
    &=-\frac{{\rm e}^{i(kz-\omega t)}}{i\omega}\left(
    \int dy\left(
    \frac{\partial^3u}{\partial x^2\partial z}+ik\frac{\partial^2u}{\partial x^2}
    \right)
    +\frac{\partial^2 u}{\partial y\partial z}+ik\frac{\partial u}{\partial y}
    \right)
  \end{split}
\end{align}
以上から
\begin{align}
  {\rm div}\vec{B}=\frac{\partial B_x}{\partial x}+\frac{\partial B_y}{\partial y}+\frac{\partial B_z}{\partial z}=0
\end{align}
を得る.
\setcounter{section}{4}
\setcounter{equation}{0}
\section*{問4(1)}
$w=w_0\sqrt{1+z^2/z_R^2}$, $z_R=kw_0^2/2$より
\begin{align}
  w(z)=\sqrt{w_0^2+\frac{4z^2}{k^2w_0^2}}
\end{align}
波長$633\ \si{\nano\metre}$のレーザー光の波数$k$は
\begin{align}
  k=\frac{2\pi}{\lambda}=9.926\times10^6\ \si{\metre^{-1}}
\end{align}
である.したがって$w_0=5\ \si{\metre}$のレーザー光が$3.8\times10^8\ \si{\metre}$進んだときのビーム径は
\begin{align}
  w=\sqrt{5^2+\frac{4\times(3.8\times10^8)^2}{(9.926\times10^6)^2\times5^2}}=16.1\ \si{\metre}
\end{align}
である.また$w_0=1\ \si{\micro\metre}$のレーザー光が$100\ \si{\metre}$進んだときのビーム径は
\begin{align}
  w=\sqrt{(1\times10^{-6})^2+\frac{4\times100^2}{(9.926\times10^6)^2\times(1\times10^{-6})^2}}=20.1\ \si{\metre}
\end{align}
である.
\section*{問4(2)}
ガウスビームの電場振幅は
\begin{align}
  u(x,y,z)=U_0\frac{q_0}{q_0+z}{\rm e}^{-(x^2+y^2)/w^2}{\rm e}^{ik(x^2+y^2)/2R}
\end{align}
である.
$z\rightarrow0$において曲率半径$R$が$\infty$となることから${\rm e}^{ik(x^2+y^2)/2R}\rightarrow 1$である.
また$w\rightarrow w_0$なので
\begin{align}
  u(x,y,0)=U_0{\rm e}^{-(x^2+y^2)/w_0^2}
\end{align}
になる.このときレーザー光の$(x,y,0)$における強度は
\begin{align}
  \frac{1}{2}c\varepsilon_0|u(x,y,0)|^2
\end{align}
である.したがってこのガウスビームのパワー$P$はガウス積分
\begin{align}
  \int dS\ {\rm e}^{-\alpha(x^2+y^2)}=\frac{\pi}{4\alpha}
\end{align}
を用いて
\begin{align}
  \begin{split}
    P&=\frac{1}{2}c\varepsilon_0\int dS\ \left|U_0{\rm e}^{-(x^2+y^2)/w_0^2}\right|^2\\
    &=\frac{c\varepsilon_0U_0^2}{2}\int dS\ {\rm e}^{-2(x^2+y^2)/w_0^2}\\
    &=\frac{c\varepsilon_0U_0^2}{2}\frac{\pi w_0^2}{8}
  \end{split}
\end{align}
となる.ここでレーザー光の半径$w_0=50\ \si{\micro\metre}$,出力$P=1\ \si{\watt}$のとき
\begin{align}
  U_0=\sqrt{\frac{16P}{\pi c\varepsilon_0w_0^2}}=8.760\times10^5\ \si{\volt.\metre^{-1}}
\end{align}
したがって電場振幅の分布は
\begin{align}
  u(x,y,0)=(8.760\times10^5){\rm e}^{-(x^2+y^2)/w_0^2}
\end{align}
したがってレーザーの中心$(x=y=0)$における電場強度は
\begin{align}
  U(0,0,0)=8.76\times10^5\ \si{\volt.\metre^{-1}}
\end{align}
である.
\section*{問4(3)}
近軸近似が成り立たないのは以下の場合である.
\subsection*{2. 1.と同じ光を焦点距離$1\ \si{cm}$のレンズで集光した}
平行光をレンズで集光する場合,光線は図\ref{fig:conv.jpg}のような形になる.
ここで$w$はビーム径, $f$は焦点距離である.
電場の大きさは, $x,y$方向については$w$程度の大きさで変化するため
\begin{align}
  \frac{\partial}{\partial x}\sim\frac{\partial}{\partial y}\sim\frac{1}{w}
\end{align}
程度と見積もることが出来る.一方で$z$方向へは$f$程度の長さで変化するため
\begin{align}
  \frac{\partial}{\partial z}\sim\frac{1}{f}
\end{align}
程度と見積もれる.今回の想定において$w=f=1\ \si{cm}$であるため
\begin{align}
  \frac{\partial^2}{\partial x^2}\sim\frac{\partial^2}{\partial y^2}\sim10^4\ \si{m^{-1}}
\end{align}
\begin{align}
  \frac{\partial^2}{\partial z^2}\sim10^4\ \si{m^{-1}}
\end{align}
と$\partial^2 u/\partial z^2$の項が$\partial^2 u/\partial x^2$と同程度の大きさになり,無視できない.
したがってこの想定では近軸近似は成立しない.
\mfig[width=8cm]{conv.jpg}{集光の模式図}
\subsection*{3. 波長$633\ \si{nm}$の点光源からの光を
焦点距離$10\ \si{cm}$, 半径$10\ \si{cm}$のレンズで平行光にした}
点光源からの光を平行光にすることは図\ref{fig:conv.jpg}において光線の向きを逆にしたものと等価であるので,同様の議論が行える.
すなわち$w=f=10\ \si{cm}$であるため
\begin{align}
  \frac{\partial^2}{\partial x^2}\sim\frac{\partial^2}{\partial y^2}\sim10^2\ \si{m^{-1}}
\end{align}
\begin{align}
  \frac{\partial^2}{\partial z^2}\sim10^2\ \si{m^{-1}}
\end{align}
よって$\partial^2 u/\partial z^2\sim\partial^2 u/\partial x^2$より,近軸近似は成立しない.
\subsection*{4. 3.と同じ光を焦点距離$1\ \si{cm}$,半径$2\ \si{mm}$のレンズで平行光にした}
同様に各項の大きさを見積もると
\begin{align}
  \frac{\partial^2}{\partial x^2}\sim\frac{\partial^2}{\partial y^2}\sim2.5\times10^5\ \si{m^{-1}}
\end{align}
\begin{align}
  \frac{\partial^2}{\partial z^2}\sim10^4\ \si{m^{-1}}
\end{align}
と1桁程度しか異ならないため,近軸近似は成立しない.
\subsection*{5. 波長$10\ \si{\micro\metre}$,ビーム径$1\ \si{mm}$の平行光を直径$10\ \si{\micro\metre}$のピンホールに通した}
ピンホールによって光線の径が変わる場合,図\ref{fig:slit.jpg}のように光線の径僅かな距離で急激に変化する.
したがって$\partial^2/\partial z^2$の項が非常に大きくなるため,近軸近似は成立しない.
\mfig[width=8cm]{slit.jpg}{スリットによる光線の変化}
\end{document}