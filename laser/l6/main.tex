\documentclass[uplatex,a4j,11pt,dvipdfmx]{jsarticle}
\usepackage{listings,jvlisting}
\bibliographystyle{junsrt}

\usepackage{url}

\usepackage{graphicx}
\usepackage{gnuplot-lua-tikz}
\usepackage{pgfplots}
\usepackage{tikz}
\usepackage{amsmath,amsfonts,amssymb}
\usepackage{bm}
\usepackage{siunitx}

\makeatletter
\def\fgcaption{\def\@captype{figure}\caption}
\makeatother
\newcommand{\setsections}[3]{
\setcounter{section}{#1}
\setcounter{subsection}{#2}
\setcounter{subsubsection}{#3}
}
\newcommand{\mfig}[3][width=15cm]{
\begin{center}
\includegraphics[#1]{#2}
\fgcaption{#3 \label{fig:#2}}
\end{center}
}
\newcommand{\gnu}[2]{
\begin{figure}[hptb]
\begin{center}
\input{#2}
\caption{#1}
\label{fig:#2}
\end{center}
\end{figure}
}

\begin{document}
\title{レーザー物理学 レポート No.5}
\author{82311971 佐々木良輔}
\date{}
\maketitle
\section*{問8(1)}
$\hbar\omega=\varepsilon$とする.分配関数は
\begin{align}
  Z=\sum_{n=0}^\infty e^{-\beta n\varepsilon}
\end{align}
なので$n\varepsilon$の期待値は
\begin{align}
  \begin{split}
    \langle\varepsilon\rangle=&\sum_{n=0}^\infty\frac{n\varepsilon e^{-\beta n\varepsilon}}{Z}\\
    =&\frac{-\frac{\partial}{\partial\beta}\sum_n e^{-\beta n\varepsilon}}{\sum_ne^{-\beta n\varepsilon}}\\
    =&-\frac{\partial}{\partial\beta}\log\left(\sum_{n=0}^\infty e^{-\beta n\varepsilon}\right)
  \end{split}
\end{align}
ここで無限等比級数の和の公式から
\begin{align}
  \begin{split}
    \sum_{n=0}^\infty e^{-\beta n\varepsilon}=\frac{1}{1-e^{-\beta\varepsilon}}
  \end{split}
\end{align}
なので
\begin{align}
  \begin{split}
    \langle\varepsilon\rangle=&-\frac{\partial}{\partial\beta}\log\left(1-e^{-\beta\varepsilon}\right)^{-1}\\
    =&\frac{\partial}{\partial\beta}\log\left(1-e^{-\beta\varepsilon}\right)\\
    =&\frac{\varepsilon e^{-\beta\varepsilon}}{1-e^{-\beta\varepsilon}}\\
    =&\frac{\varepsilon}{e^{\beta\varepsilon}-1}
  \end{split}
\end{align}
を得る.
\section*{問8(2)}
スライド(2.5)式について$\omega=2\pi\nu$を用いて, (2.4)式は
\begin{align}
  \begin{array}{cl}
    &\rho(\omega)d\omega=\cfrac{\hbar\omega^3}{c^3\pi^2}\cfrac{1}{e^{\beta\hbar\omega}-1}d\omega\\
    \iff&\rho(\nu)d\nu=\cfrac{\hbar(2\pi\nu)^3}{c^3\pi^2}\cfrac{1}{e^{\beta\hbar\times2\pi\nu}-1}2\pi d\nu\\
    \iff&\rho(\nu)d\nu=\cfrac{8\pi h\nu^3}{c^3}\cfrac{1}{e^{\beta h\nu}-1}d\nu
  \end{array}
\end{align}
となる.ただし$h$はプランク定数である.また(2.6)式について$c=\nu\lambda$から
\begin{align}
  \frac{d\nu}{d\lambda}=\frac{d}{d\lambda}\frac{c}{\lambda}=-\frac{c}{\lambda^2}
\end{align}
であるので, (5)式より
\begin{align}
  \begin{array}{cl}
    &\rho(\nu)d\nu=\cfrac{8\pi h\nu^3}{c^3}\cfrac{1}{e^{\beta h\nu}-1}d\nu\\
    \iff&\rho(\lambda)d\lambda=\cfrac{8\pi h}{\lambda^3}\cfrac{1}{e^{\beta hc/\lambda}-1}\left(-\cfrac{c}{\lambda^2}d\lambda\right)\\
    \iff&\rho(\lambda)d\lambda=-\cfrac{8\pi hc}{\lambda^5}\cfrac{1}{e^{\beta hc/\lambda}-1}d\lambda\\
  \end{array}
\end{align}
となる.
\bibliography{ref.bib}
\end{document}