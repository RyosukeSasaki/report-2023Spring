\documentclass[uplatex,a4j,11pt,dvipdfmx]{jsarticle}
\usepackage{listings,jvlisting}
\bibliographystyle{junsrt}

\usepackage{url}

\usepackage{graphicx}
\usepackage{gnuplot-lua-tikz}
\usepackage{pgfplots}
\usepackage{tikz}
\usepackage{amsmath,amsfonts,amssymb}
\usepackage{bm}
\usepackage{siunitx}

\makeatletter
\renewcommand{\theequation}{%
\thesection.\arabic{equation}}
\@addtoreset{equation}{section}
\makeatother

\makeatletter
\def\fgcaption{\def\@captype{figure}\caption}
\makeatother
\newcommand{\setsections}[3]{
\setcounter{section}{#1}
\setcounter{subsection}{#2}
\setcounter{subsubsection}{#3}
}
\newcommand{\mfig}[3][width=15cm]{
\begin{center}
\includegraphics[#1]{#2}
\fgcaption{#3 \label{fig:#2}}
\end{center}
}
\newcommand{\gnu}[2]{
\begin{figure}[hptb]
\begin{center}
\input{#2}
\caption{#1}
\label{fig:#2}
\end{center}
\end{figure}
}

\begin{document}
\title{レーザー物理学 レポート No.6}
\author{82311971 佐々木良輔}
\date{}
\maketitle
\section*{問14}
\setcounter{section}{14}
\setcounter{equation}{0}
$A=(N_2-N_1)|\mu_{12}|^2/\hbar\varepsilon_0$, $\gamma_2\Omega_0^2/\gamma_1\ll 1$とすると
\begin{align}
  \begin{split}
    \chi(\omega)&=A\frac{\delta-i\gamma_2}{\delta^2+\gamma_2^2}\\
    &=A\frac{1}{\delta+i\gamma_2}\\
  \end{split}
\end{align}
$A$は定数なので,簡単のため以下では$A=1$とする.
$\delta=\omega-\omega_0=x+iy$とすると
\begin{align}
  \begin{split}
    \chi(\omega)&=\frac{1}{x+i(y+\gamma_2)}\\
    &=\frac{x-i(y+\gamma)}{x^2+(y+\gamma_2)^2}\\
    &=:u(x,y)+iv(x,y)
  \end{split}
\end{align}
ここで
\begin{align}
  \begin{split}
    u(x,y)=\frac{x}{x^2+(y+\gamma_2)^2}\\
    v(x,y)=\frac{-y-\gamma_2}{x^2+(y+\gamma_2)^2}
  \end{split}
\end{align}
とした.このとき
\begin{align}
  \frac{du}{dx}=\frac{(y+\gamma_2)^2-x^2}{\left(x^2+(y+\gamma_2)^2\right)^2}=\frac{dv}{dy}\\
  \frac{du}{dy}=\frac{-2x(y+\gamma_2)}{\left(x^2+(y+\gamma_2)^2\right)^2}=-\frac{dv}{dx}
\end{align}
よりコーシー・リーマンの関係式を満たすため$\chi(\omega)$は正則である.
また$x+i(y+\gamma)=re^{i\theta}$とすると(14.2)から
\begin{align}
  \chi(\omega)=\frac{re^{-i\theta}}{r^2}\rightarrow 0\ (r\rightarrow\infty)
\end{align}
である.このとき図1のような積分経路において以下の積分を行う
\begin{align}
  \int_Cd\omega'\frac{\chi(\omega')}{\omega'-\omega}=\left(\int_{-R}^{\omega-r}+\int_{C_1}+\int_{\omega+r}^{R}+\int_{C_2}\right)d\omega'\frac{\chi(\omega')}{\omega'-\omega}
\end{align}
ここで(14.6)から$R\rightarrow\infty$において$C_2$上の積分は$0$である.また$C_1$上の積分は留数定理から
\begin{align}
  \int_{C_1}d\omega'\frac{\chi(\omega')}{\omega'-\omega}=-i\pi\chi(\omega)
\end{align}
である.したがって
\begin{align}
  \begin{split}
    \lim_{\varepsilon\rightarrow0}\left(\int_{-\infty}^{\omega-r}+\int_{\omega+r}^{\infty}\right)d\omega'\frac{\chi(\omega')}{\omega'-\omega}=
    \mathcal{P}\int_{-\infty}^\infty d\omega'\frac{\chi(\omega')}{\omega'-\omega}=i\pi\chi(\omega)
  \end{split}\nonumber\\
  \begin{split}
    \iff -\mathcal{P}\int_{-\infty}^\infty \frac{id\omega'}{\pi}\frac{\chi(\omega')}{\omega'-\omega}=\chi(\omega)
  \end{split}
\end{align}
$\chi(\omega)=\chi'(\omega)-i\chi''(\omega)$より
\begin{align}
  -\mathcal{P}\int_{-\infty}^{\infty}\frac{d\omega'}{\pi}\frac{i\chi(\omega')+\chi''(\omega')}{\omega'-\omega}=\chi'(\omega)-i\chi''(\omega)
\end{align}
両辺の実部と虚部を比較し
\begin{align}
  \chi'(\omega)&=-\mathcal{P}\int_{-\infty}^{\infty}\frac{d\omega'}{\pi}\frac{\chi''(\omega')}{\omega'-\omega}\\
  \chi''(\omega)&=\mathcal{P}\int_{-\infty}^{\infty}\frac{d\omega'}{\pi}\frac{\chi'(\omega')}{\omega'-\omega}
\end{align}
を得る.
\mfig[width=8cm]{14.jpg}{積分経路}
\section*{問15}
\subsection*{(1)}
\setcounter{section}{15}
\setcounter{equation}{0}
速度分布は
\begin{align}
  \rho(v_x,v_y,v_z)=A\exp\left(-\frac{m(v_x^2+v_y^2+v_z^2)}{2kT}\right)=A\exp\left(-\frac{mv^2}{2kT}\right)
\end{align}
である.ここで定数$A$は規格化条件から
\begin{align}
  \begin{split}
    1=A\int dv_x e^{-mv_x^2/2kT}\int dv_y e^{-mv_y^2/2kT}\int dv_z e^{-mv_z^2/2kT}=A\left(\int dv_x e^{-mv_x^2/2kT}\right)^3
  \end{split}
\end{align}
を満たす.ガウス積分$\int e^{-\alpha x^2}dx=\sqrt{\pi/\alpha}$から
\begin{align}
  A=\left(\int dv_x e^{-mv_x^2/2kT}\right)^{-3}=\left(\sqrt{\frac{2\pi kT}{m}}\right)^{-3}=\left(\frac{m}{2\pi kT}\right)^{3/2}
\end{align}
となる.

速さ$v$の粒子を見出す確率は$\rho$を$v$から$v+dv$の範囲で積分したものである.
$dv$が十分小さければ,この積分は$v$から$v+dv$の球殻の体積をかけることに等しいため
\begin{align}
  \rho(v_x,v_y,v_z)dv=A\exp\left(-\frac{mv^2}{2kT}\right)4\pi v^2dv
\end{align}
となる.これが最大となることから
\begin{align*}
    \frac{d\rho}{dv}&=2vA\exp\left(-\frac{mv^2}{2kT}\right)-v^2\frac{mv}{kT}A\exp\left(-\frac{mv^2}{2kT}\right)\\
    &=A\left(2-\frac{mv^2}{kT}\right)v\exp\left(-\frac{mv^2}{2kT}\right)=0
\end{align*}
\begin{align}
  \begin{split}
    \begin{array}{cc}
      \iff&\cfrac{mv^2}{kT}=2\\
      \iff&v=\sqrt{\cfrac{2kT}{m}}
    \end{array}
  \end{split}
\end{align}
を得る.
\subsection*{(2)}
速さ$v$の期待値は
\begin{align}
  \overline{v}=\int_0^\infty v\cdot 4\pi Av^2\exp\left(-\frac{mv^2}{2kT}\right)dv
\end{align}
ここでガウス積分$\int_0^\infty x^3e^{-\alpha x^2}=1/2\alpha^2$を用いて
\begin{align}
  \begin{split}
    \overline{v}&=4\pi A\int_0^\infty v^3\exp\left(-\frac{mv^2}{2kT}\right)dv\\
    &=4\pi\left(\frac{m}{2\pi kT}\right)^{3/2}\frac{1}{2}\left(\frac{2kT}{m}\right)^2\\
    &=\sqrt{\frac{8kT}{\pi m}}
  \end{split}
\end{align}
となる.次に二乗平均速さは同様にガウス積分$\int_0^\infty x^4e^{-\alpha x^2}=3/8\sqrt{\pi/\alpha^5}$を用いて
\begin{align}
  \begin{split}
    \overline{v^2}&=\int_0^\infty v^2\cdot4\pi Av^2\exp\left(-\frac{mv^2}{2kT}\right)dv\\
    &=4\pi\left(\frac{m}{2\pi kT}\right)^{3/2}\frac{3}{8}\sqrt{\frac{32\pi k^5T^5}{m^5}}\\
    &=\frac{3kT}{m}
  \end{split}
\end{align}
となる.
\section*{問16}
\setcounter{section}{16}
\setcounter{equation}{0}
\bibliography{ref.bib}
\end{document}