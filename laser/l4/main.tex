\documentclass[uplatex,a4j,11pt,dvipdfmx]{jsarticle}
\usepackage{listings,jvlisting}
\bibliographystyle{junsrt}

\usepackage{url}

\usepackage{graphicx}
\usepackage{gnuplot-lua-tikz}
\usepackage{pgfplots}
\usepackage{tikz}
\usepackage{amsmath,amsfonts,amssymb}
\usepackage{bm}
\usepackage{siunitx}

\makeatletter
\def\fgcaption{\def\@captype{figure}\caption}
\makeatother
\newcommand{\setsections}[3]{
\setcounter{section}{#1}
\setcounter{subsection}{#2}
\setcounter{subsubsection}{#3}
}
\newcommand{\mfig}[3][width=15cm]{
\begin{center}
\includegraphics[#1]{#2}
\fgcaption{#3 \label{fig:#2}}
\end{center}
}
\newcommand{\gnu}[2]{
\begin{figure}[hptb]
\begin{center}
\input{#2}
\caption{#1}
\label{fig:#2}
\end{center}
\end{figure}
}

\begin{document}
\title{レーザー物理学 レポート No.3}
\author{82311971 佐々木良輔}
\date{}
\maketitle
\section*{問5(1)}
凸面と凹面の曲率が等しいことから$R_1=R_2=R>0$, したがって光線行列は
\begin{align}
  \begin{split}
    &\left(
      \begin{array}[2]{cc}
        1-\cfrac{L(n_2-n_1)}{n_2R_1}&\cfrac{L}{n_2}\\
        (n_2-n_1)\left(\cfrac{1}{R_2}-\cfrac{1}{R_1}\right)-\cfrac{L(n_2-n_1)^2}{n_2R_1R_2}&1+\cfrac{L(n_2-n_1)}{n_2R_2}
      \end{array}
    \right)\\
    =&\left(
      \begin{array}[2]{cc}
        1-\cfrac{L(n_2-n_1)}{n_2R}&\cfrac{L}{n_2}\\
        -\cfrac{L(n_2-n_1)^2}{n_2R^2}&1+\cfrac{L(n_2-n_1)}{n_2R}
      \end{array}
    \right)
  \end{split}
\end{align}
この光学素子に$w_1$, $\theta=0$の平行光を入射したとき,出射光$w_2$, $\theta$は
\begin{align*}
  \renewcommand\arraystretch{2}
  \begin{split}
    \begin{pmatrix}
      w_2\\n_1\theta
    \end{pmatrix}
    &=
    \begin{pmatrix}
      1-\cfrac{L(n_2-n_1)}{n_2R}&\cfrac{L}{n_2}\\
      -\cfrac{L(n_2-n_1)^2}{n_2R^2}&1+\cfrac{L(n_2-n_1)}{n_2R}
    \end{pmatrix}
    \begin{pmatrix}
      w_1\\0
    \end{pmatrix}
  \end{split}
\end{align*}
\begin{align}
  \begin{split}
    \therefore\ w_2&=w_1\left(1-\cfrac{L(n_2-n_1)}{n_2R}\right)\\
    \theta&=-w_1\frac{L(n_2-n_1)^2}{n_2n_1R^2}
  \end{split}
\end{align}
ここで図\ref{fig:laser_04_q51.pdf}のようにメニスカスレンズの出射面を$x=0$におくと,出射光は
\begin{align}
  y=w_2+\theta x
\end{align}
という直線で表される.ただし$w_1$が大きい場合は図\ref{fig:laser_04_q51.pdf}の黄点線のようにレンズの曲率による変位$\delta$が大きくなるので,
ここでは$w_1$が十分小さいものとする.
ここで$x$切片は
\begin{align}
  \begin{split}
    0&=w_2+\theta x\\
    0&=w_1\left(1-\cfrac{L(n_2-n_1)}{n_2R}-\cfrac{L(n_2-n_1)^2}{n_2n_1R^2}x\right)\\
    \therefore\ x&=\cfrac{n_2n_1R^2}{L(n_2-n_1)^2}\left(1-\cfrac{L(n_2-n_1)}{n_2R}\right)
  \end{split}
\end{align}
となり$x$切片は定数となる.
ここで$1-L(n_2-n_1)/n_2R\geq0$のとき, $w_1$によらず平行光は正の$x$座標で収束することから,メニスカスレンズは凸レンズとして機能する.
一方で$1-L(n_2-n_1)/n_2R<0$のとき, $w_1$によらず平行光は負の$x$座標で収束することから,メニスカスレンズは凹レンズとして機能する.
\mfig[width=12cm]{laser_04_q51.pdf}{メニスカスレンズに平行光を入れた場合の模式図}
\section*{問5(2)}
共焦点共振器では焦点が互いの鏡の中心にあることから,平行光は図\ref{fig:laser_04_q52.pdf} ①のように
向かいの鏡の中心に向けて反射する.次に②のように鏡の中心で反射した光線は上下対称に反射するため,③の反射で再び平行光に戻る.
③で反射された平行光は④で再び焦点を通るように反射する.以降同様の反射を永遠に繰り返すことができる.
\mfig[width=10cm]{laser_04_q52.pdf}{共焦点共振器の模式図}
\section*{問5(3)}
以下では$n=1$とする.
共振器内を一往復するときの光線行列は
\begin{align}
  \begin{split}
    &\begin{pmatrix}
      1&0\\
      -2/R_1&-1
    \end{pmatrix}
    \begin{pmatrix}
      1&-L\\
      0&1
    \end{pmatrix}
    \begin{pmatrix}
      1&0\\
      -2/R_2&-1
    \end{pmatrix}
    \begin{pmatrix}
      1&L\\
      0&1
    \end{pmatrix}\\
    =&\begin{pmatrix}
      1&0\\
      -2/R_1&-1
    \end{pmatrix}
    \begin{pmatrix}
      1&-L\\
      0&1
    \end{pmatrix}
    \begin{pmatrix}
      1&L\\
      -2/R_2&-2L/R_2-1
    \end{pmatrix}\\
    =&\frac{1}{R_2}\begin{pmatrix}
      1&0\\
      -2/R_1&-1
    \end{pmatrix}
    \begin{pmatrix}
      R_2+2L&2(R_2+L)L\\
      -2&-2L-R_2
    \end{pmatrix}\\
    =&\frac{1}{R_2}\begin{pmatrix}
      R_2+2L&2(R_2+L)L\\
      -2\cfrac{R_2-R_1+2L}{R_1}&-\cfrac{4L^2+4R_2L-2R_1L-R_1R_2}{R_1}
    \end{pmatrix}
  \end{split}
\end{align}
またABCD則において,ガウスビームが安定であるためには$q'=q$であるべきなので
\begin{eqnarray}
  &q=\cfrac{Aq+B}{Cq+D}\nonumber\\
  \iff&0=B\left(\frac{1}{q}\right)^2+\frac{1}{q}(A-D)-C\nonumber\\
  \iff&\cfrac{1}{q}=\frac{D-A\pm\sqrt{(A-D)^2+4BC}}{2B}
\end{eqnarray}
ここで$A,B,C,D$それぞれは実数であるが, $1/q$が複素数であるべきなので$(A-D)^2+4BC<0$とする.
ここで
\begin{align}
  \frac{1}{q}=\frac{1}{R}+\frac{2i}{kw^2}
\end{align}
より
\begin{align}
  \frac{D-A}{2B}&=\frac{1}{R}\\
  \pm\sqrt{\left(\frac{A-D}{2B}\right)^2+\frac{C}{B}}&=\frac{2i}{kw^2}
\end{align}
となる.ただし$R$はミラー1上でのガウスビームの曲率, $k$は波数, $w$はビーム径である.
(8)式から
\begin{align}
  \begin{split}
    \frac{1}{R}&=\frac{1}{4(R_2+L)L}\left(-\cfrac{4L^2+4R_2L-2R_1L-R_1R_2}{R_1}-(R_2+2L)\right)\\
    &=\frac{1}{4(R_2+L)L}\left(-\cfrac{4L^2+4R_2L-2R_1L-R_1R_2}{R_1}-\frac{R_1R_2+2R_1L}{R_1}\right)\\
    &=\frac{1}{4(R_2+L)L}\frac{-4(L+R_2)L}{R_1}\\
    &=\frac{1}{-R_1}
  \end{split}
\end{align}
となる.また(9)式から
\begin{eqnarray}
  &\begin{split}
    \frac{2i}{kw^2}&=\pm\sqrt{\left(\frac{A-D}{2B}\right)^2+\frac{C}{B}}\\
    &=\pm\sqrt{\frac{1}{R_1^2}-\frac{R_2-R_1+2L}{R_1(R_2+L)L}}
  \end{split}\nonumber\\
  \iff&\frac{2}{kw^2}=\pm\sqrt{\frac{R_2-R_1+2L}{R_1(R_2+L)L}-\frac{1}{R_1^2}}
\end{eqnarray}
両辺の符号が一致すべきなので
\begin{eqnarray}
  &\frac{2}{kw^2}=\sqrt{\frac{R_2-R_1+2L}{R_1(R_2+L)L}-\frac{1}{R_1^2}}\nonumber\\
  \iff&w^2=\frac{2}{k\sqrt{\frac{R_2-R_1+2L}{R_1(R_2+L)L}-\frac{1}{R_1^2}}}\nonumber\\
  \iff&w=\sqrt{\frac{2}{k\sqrt{\frac{R_2-R_1+2L}{R_1(R_2+L)L}-\frac{1}{R_1^2}}}}
\end{eqnarray}
\bibliography{ref.bib}
となる.以上からミラー1上でのガウスビームの曲率及びビーム径は
\begin{align}
  \begin{split}
    |R|&=R_1\\
    w&=\sqrt{\frac{2}{k\sqrt{\frac{R_2-R_1+2L}{R_1(R_2+L)L}-\frac{1}{R_1^2}}}}
  \end{split}
\end{align}
となる.
\end{document}