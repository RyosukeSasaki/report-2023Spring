\documentclass[uplatex,a4j,11pt,dvipdfmx]{jsarticle}
\usepackage{listings,jvlisting}
\bibliographystyle{junsrt}

\usepackage{url}

\usepackage{graphicx}
\usepackage{gnuplot-lua-tikz}
\usepackage{pgfplots}
\usepackage{tikz}
\usepackage{amsmath,amsfonts,amssymb}
\usepackage{bm}
\usepackage{siunitx}

\makeatletter
\renewcommand{\theequation}{%
\thesection.\arabic{equation}}
\@addtoreset{equation}{section}
\makeatother

\makeatletter
\def\fgcaption{\def\@captype{figure}\caption}
\makeatother
\newcommand{\setsections}[3]{
\setcounter{section}{#1}
\setcounter{subsection}{#2}
\setcounter{subsubsection}{#3}
}
\newcommand{\mfig}[3][width=15cm]{
\begin{center}
\includegraphics[#1]{#2}
\fgcaption{#3 \label{fig:#2}}
\end{center}
}
\newcommand{\gnu}[2]{
\begin{figure}[hptb]
\begin{center}
\input{#2}
\caption{#1}
\label{fig:#2}
\end{center}
\end{figure}
}

\begin{document}
\title{レーザー物理学 レポート No.6}
\author{82311971 佐々木良輔}
\date{}
\maketitle
\section*{問11}
\setcounter{section}{11}
\setcounter{equation}{0}
(2.30)式の左辺は
\begin{align}
  \begin{split}
    \frac{d\rho}{dt}&=\frac{d}{dt}\left(|\psi\rangle\langle\psi|\right)\\
    &=\left(\frac{d}{dt}|\psi\rangle\right)\langle\psi|+|\psi\rangle\frac{d}{dt}\langle\psi|
  \end{split}
\end{align}
ここでSchr\"{o}dinger方程式から
\begin{align}
  \frac{d}{dt}|\psi\rangle&=-\frac{i}{\hbar}H|\psi\rangle\\
  \frac{d}{dt}\langle\psi|&=\frac{i}{\hbar}\langle\psi|H
\end{align}
を用いて(11.1)は
\begin{align}
  \begin{split}
    \frac{d\rho}{dt}&=-\frac{i}{\hbar}H|\psi\rangle\langle\psi|+\frac{i}{\hbar}|\psi\rangle\langle\psi|H\\
    &=-\frac{i}{\hbar}H\rho+\frac{i}{\hbar}\rho H\\
    &=\frac{i}{\hbar}[\rho,H]
  \end{split}
\end{align}
となる.
\newpage
\section*{問12}
$\mu_{12}E_0^*/\hbar=\Omega$とすると
\begin{align}
  \begin{pmatrix}
    \dot{\rho_{11}}\\
    \dot{\rho_{22}}\\
    \dot{\rho_{21}}\\
    \dot{\rho_{12}}\\
  \end{pmatrix}=\frac{i}{2}
  \begin{pmatrix}
    0&0&\Omega e^{i\delta t}&-\Omega^* e^{-i\delta t}\\
    0&0&-\Omega e^{i\delta t}&\Omega^* e^{-i\delta t}\\
    \Omega^* e^{-i\delta t}&-\Omega^* e^{-i\delta t}&0&0\\
    -\Omega e^{i\delta t}&\Omega e^{i\delta t}&0&0\\
  \end{pmatrix}
  \begin{pmatrix}
    \rho_{11}\\
    \rho_{22}\\
    \rho_{21}\\
    \rho_{12}\\
  \end{pmatrix}
\end{align}
この係数行列について,固有値を求める
\begin{align}
  \begin{split}
    \left|
    \begin{array}{cccc}
      -\lambda&0&\Omega e^{i\delta t}&-\Omega^* e^{-i\delta t}\\
      0&-\lambda&-\Omega e^{i\delta t}&\Omega^* e^{-i\delta t}\\
      \Omega^* e^{-i\delta t}&-\Omega^* e^{-i\delta t}&-\lambda&0\\
      -\Omega e^{i\delta t}&\Omega e^{i\delta t}&0&-\lambda\\
    \end{array}
    \right|
    =&-\lambda
    \left|
    \begin{array}{ccc}
      -\lambda&-\Omega e^{i\delta t}&\Omega^* e^{-i\delta t}\\
      -\Omega^* e^{-i\delta t}&-\lambda&0\\
      \Omega e^{i\delta t}&0&-\lambda\\
    \end{array}
    \right|\\
    &+\Omega^* e^{-i\delta t}
    \left|
    \begin{array}{cccc}
      0&\Omega e^{i\delta t}&-\Omega^* e^{-i\delta t}\\
      -\lambda&-\Omega e^{i\delta t}&\Omega^* e^{-i\delta t}\\
      \Omega e^{i\delta t}&0&-\lambda\\
    \end{array}
    \right|\\
    &+\Omega e^{i\delta t}
    \left|
    \begin{array}{cccc}
      0&\Omega e^{i\delta t}&-\Omega^* e^{-i\delta t}\\
      -\lambda&-\Omega e^{i\delta t}&\Omega^* e^{-i\delta t}\\
      -\Omega^* e^{-i\delta t}&-\lambda&0\\
    \end{array}
    \right|\\
    =&\lambda^2(\lambda^2-4|\Omega|^2)
  \end{split}
\end{align}
よって固有値は$\lambda=0,\pm2|\Omega|$である.
\newpage
\section*{問13}
\setcounter{section}{13}
\setcounter{equation}{0}
$[\rho,H]$を計算する.
\begin{align}
  \begin{split}
    [\rho,H]&=
    \begin{pmatrix}
      \rho_{11}&\rho_{12}\\\rho_{21}&\rho_{22}
    \end{pmatrix}
    \begin{pmatrix}
      H_{11}&H_{12}\\H_{21}&H_{22}
    \end{pmatrix}
    -
    \begin{pmatrix}
      H_{11}&H_{12}\\H_{21}&H_{22}
    \end{pmatrix}
    \begin{pmatrix}
      \rho_{11}&\rho_{12}\\\rho_{21}&\rho_{22}
    \end{pmatrix}\\
    &=
    \begin{pmatrix}
      \rho_{12}H_{21}-\rho_{21}H_{12}&\rho_{11}H_{12}+\rho_{12}H_{22}-H_{11}\rho_{12}-H_{12}\rho_{22}\\
      \rho_{21}H_{11}+\rho_{22}H_{21}-H_{21}\rho_{11}-H_{22}\rho_{21}&\rho_{21}H_{12}-H_{21}\rho_{12}
    \end{pmatrix}
  \end{split}
\end{align}
したがってSchr\"{o}dinger方程式は
\begin{align}
  \begin{pmatrix}
    \dot{\rho_{11}}&\dot{\rho_{12}}\\
    \dot{\rho_{21}}&\dot{\rho_{22}}
  \end{pmatrix}
  =\frac{i}{\hbar}
  \begin{pmatrix}
    \rho_{12}H_{21}-\rho_{21}H_{12}&\rho_{11}H_{12}+\rho_{12}H_{22}-H_{11}\rho_{12}-H_{12}\rho_{22}\\
    \rho_{21}H_{11}+\rho_{22}H_{21}-H_{21}\rho_{11}-H_{22}\rho_{21}&\rho_{21}H_{12}-H_{21}\rho_{12}
  \end{pmatrix}
\end{align}
この行列の22成分から11成分を引くと
\begin{align}
  \dot{\rho_{22}}-\dot{\rho_{11}}=\frac{i}{\hbar}(2\rho_{21}H_{12}-2H_{21}\rho_{12})
\end{align}
ここで左辺は$\dot{\rho_{22}}-\dot{\rho_{11}}=\dot{r_3}$である.また
\begin{align}
  \begin{split}
    ({\bm F}\times {\bm r})_3&=F_1r_2-F_2r_1\\
    &=\frac{i}{\hbar}(H_{12}+H_{21})(\rho_{21}-\rho_{12})+\frac{i}{\hbar}(H_{12}-H_{21})(\rho_{12}+\rho_{21})\\
    &=\frac{i}{\hbar}(2H_{12}\rho_{21}-2H_{21}\rho_{12})
  \end{split}
\end{align}
である. (13.3), (13.4)から
\begin{align}
  \dot{r_3}=({\bm F}\times {\bm r})_3
\end{align}
である.次に(13.2)の12成分と21成分を足すと
\begin{align}
  \begin{split}
    \dot{\rho_{12}}+\dot{\rho_{21}}=&\frac{i}{\hbar}\left(\rho_{11}H_{12}-\rho_{11}H_{21}+\rho_{12}H_{22}-\rho_{12}H_{11}\right.\\
    &+\left.\rho_{21}H_{11}-\rho_{21}H_{22}+\rho_{22}H_{21}-\rho_{22}H_{12}\right)\\
    =&\frac{i}{\hbar}(H_{21}-H_{12})(\rho_{22}-\rho_{11})+\frac{i}{\hbar}(H_{22}-H_{11})(\rho_{12}-\rho_{21})\\
    =&(F_2r_3-F_3r_2)=({\bm F}\times{\bm r})_1
  \end{split}
\end{align}
また$\dot{\rho_{12}}+\dot{\rho_{21}}=\dot{r_1}$より
\begin{align}
  \dot{r_1}=({\bm F}\times{\bm r})_1
\end{align}
である.次に(13.2)の12成分と21成分を引くと
\begin{align}
  \begin{split}
    \dot{\rho_{12}}-\dot{\rho_{21}}=&\frac{i}{\hbar}
    \left(\rho_{11}H_{12}+\rho_{11}H_{21}+\rho_{12}H_{22}-\rho_{12}H_{11}\right.\\
    &-\left.\rho_{21}H_{11}+\rho_{21}H_{22}-\rho_{22}H_{21}-\rho_{22}H_{12}\right)\\
    =&\frac{i}{\hbar}(H_{22}-H_{11})(\rho_{12}+\rho_{21})-\frac{i}{\hbar}(H_{21}+H_{12})(\rho_{22}-\rho_{11})\\
    =&i(F_3r_1-F_1r_3)=i({\bm F}\times{\bm r})_2
  \end{split}
\end{align}
また$(\dot{\rho_{12}}-\dot{\rho_{21}})/i=\dot{r_2}$より
\begin{align}
  \dot{r_2}=({\bm F}\times{\bm r})_2
\end{align}
以上から
\begin{align}
  \frac{d}{dt}{\bm r}={\bm F}\times{\bm r}
\end{align}
が示された.
\bibliography{ref.bib}
\end{document}