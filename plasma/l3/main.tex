\documentclass[uplatex,a4j,11pt,dvipdfmx]{jsarticle}
\usepackage{listings,jvlisting}
\bibliographystyle{junsrt}

\usepackage{url}

\usepackage{graphicx}
\usepackage{gnuplot-lua-tikz}
\usepackage{pgfplots}
\usepackage{tikz}
\usepackage{amsmath,amsfonts,amssymb}
\usepackage{bm}
\usepackage{siunitx}

\makeatletter
\def\fgcaption{\def\@captype{figure}\caption}
\makeatother
\newcommand{\setsections}[3]{
\setcounter{section}{#1}
\setcounter{subsection}{#2}
\setcounter{subsubsection}{#3}
}
\newcommand{\mfig}[3][width=15cm]{
\begin{center}
\includegraphics[#1]{#2}
\fgcaption{#3 \label{fig:#2}}
\end{center}
}
\newcommand{\gnu}[2]{
\begin{figure}[hptb]
\begin{center}
\input{#2}
\caption{#1}
\label{fig:#2}
\end{center}
\end{figure}
}

\begin{document}
\title{応用プラズマ工学}
\author{82311971 佐々木良輔}
\date{}
\maketitle
\section*{問1}
温度$10\ \si{eV}$の電子の熱速度はボルツマン定数を$1.602\times10^{-19}\ \si{J.eV^{-1}}$とすると
\begin{align}
  v_{\rm th}=\sqrt{\frac{2k_BT}{m_e}}=\sqrt{\frac{2\times1.602\times10^{-19}\times10}{9.109\times10^{-31}}}=1.87\times10^6\ \si{m.s^{-1}}
\end{align}
またグラフから$10\ \si{eV}$の速度係数は$\langle\sigma_{\rm en}v\rangle=5\times10^{-9}\ \si{cm^3.s^{-1}}=5\times10^{-15}\ \si{m^3.s^{-1}}$なので
衝突周波数の平均値は
\begin{align}
  \langle\nu_{\rm en}\rangle=n_{\rm n}\langle\sigma_{\rm en}v\rangle=1.0\times10^{20}\times5\times10^{-15}=5\times10^5\ \si{Hz}
\end{align}
である.
したがって平均自由行程$\lambda_{\rm en}$は
\begin{align}
  \lambda_{\rm en}=\frac{v_{\rm th}}{\langle\nu_{\rm en}\rangle}=\frac{1.87\times10^6}{5\times10^5}=4\ \si{m}
\end{align}
である.また1回の衝突で1個の水素イオンが生成されるとすると,単位体積,単位時間あたりの水素イオン生成数は
\begin{align}
  n_e\langle\nu_{\rm en}\rangle=1.0\times10^{19}\times5\times10^5=5\times10^{24}\ \si{m^{-3}.s^{-1}}
\end{align}
である.
\section*{問2}
速度係数は
\begin{align}
  \langle\sigma_{\rm en}v\rangle=\frac{1}{n_e}\int_0^\infty dv\sigma_{\rm en}vF_e(v)
\end{align}
と定義される.ここですべての電子が一定の速度$v_0$を持っている場合
\begin{align}
  F_e(v)=n_e\delta(v-v_0)
\end{align}
であるので
\begin{align}
  \langle\sigma_{\rm en}v\rangle=\sigma_{\rm en}(v_0)v_0
\end{align}
となり, $\sigma_{\rm en}(v_0)=0$の場合は$\langle\sigma_{\rm en}v\rangle=0$となる.

一方で温度$T$が与えられたとき,気体の速度はMaxwell-Boltzmann分布
\begin{align}
  F_e(v)\propto v^2{\rm exp}\left(-\frac{1}{2}\frac{m_ev^2}{k_BT}\right)
\end{align}
に従う.これは$v_{\rm th}=\sqrt{2k_BT/m_e}$で極大を取るが,
$v_{\rm th}$以上の速度においても有限の値をもっている.
したがって電子気体の温度が電離エネルギー以下であっても,
それ以上のエネルギーを持った電子は存在しており,これが反応に寄与していると考えられる.
\section*{問3}
\subsection*{(1)}
エネルギー保存則は
\begin{align}
  \frac{1}{2}m_1v_1^2=\frac{1}{2}m_1v_1'^2+\frac{1}{2}m_2v_2'^2+\Delta U
\end{align}
運動量保存則は
\begin{align}
  m_1v_1=m_1v_1'+m_2v_2'
\end{align}
である.
\subsection*{(2)}
(9)から
\begin{align}
  2\Delta U=m_1v_1^2-m_1v_1'^2-m_2v_2'^2
\end{align}
(10)を代入すると
\begin{align}
  \begin{split}
    2\Delta U=&m_1v_1^2-m_1v_1'^2-m_2\left(\frac{m_1(v_1-v_1')}{m_2}\right)^2\\
    =&m_1v_1^2-m_1v_1'^2-\frac{m_1^2}{m_2}\left(v_1-v_1'\right)^2
  \end{split}
\end{align}
となる.
\subsection*{(3)}
(12)を$v_1'$について整理すると
\begin{align}
  \begin{split}
    2\Delta U=&m_1v_1^2-m_1v_1'^2-\frac{m_1^2}{m_2}\left(v_1^2-2v_1v_1'+v_1'^2\right)\\
    =&-v_1'^2\left(m_1+\frac{m_1^2}{m_2}\right)+\frac{2m_1^2}{m_2}v_1v_1'+v_1^2\left(m_1-\frac{m_1^2}{m_2}\right)\\
    =&-\left(m_1+\frac{m_1^2}{m_2}\right)\left(v_1'-\frac{m_1v_1}{m_1+m_2}\right)^2
    +\frac{m_2}{m_1+m_2}m_1v_1^2
  \end{split}
\end{align}
となる.したがって$\Delta U$は
\begin{align}
  v_1'=\frac{m_1v_1}{m_1+m_2}
\end{align}
のときに最大値を取り,その値は
\begin{align}
  \Delta U_{\rm max}=\frac{m_2}{m_1+m_2}\frac{1}{2}m_1v_1^2
\end{align}
となる.
\subsection*{(4)}
$m_1\ll m_2$のとき
\begin{align}
  \lim_{m_1\rightarrow0} \Delta U_{\rm max}=\frac{1}{2}m_1v_1^2
\end{align}
であり,入射粒子のエネルギーがすべて内部エネルギーになることがわかる.
一方で$m_1\simeq m_2$のとき
\begin{align}
  \Delta U_{\rm max}=\frac{1}{2}\frac{1}{2}m_1v_1^2
\end{align}
であり,入射粒子のエネルギーの半分が内部エネルギーになっている.
\bibliography{ref.bib}
\end{document}