\documentclass[uplatex,a4j,11pt,dvipdfmx]{jsarticle}
\usepackage{listings,jvlisting}
\bibliographystyle{junsrt}

\usepackage{url}

\usepackage{graphicx}
\usepackage{gnuplot-lua-tikz}
\usepackage{pgfplots}
\usepackage{tikz}
\usepackage{amsmath,amsfonts,amssymb}
\usepackage{bm}
\usepackage{siunitx}

\makeatletter
\def\fgcaption{\def\@captype{figure}\caption}
\makeatother
\newcommand{\setsections}[3]{
\setcounter{section}{#1}
\setcounter{subsection}{#2}
\setcounter{subsubsection}{#3}
}
\newcommand{\mfig}[3][width=15cm]{
\begin{center}
\includegraphics[#1]{#2}
\fgcaption{#3 \label{fig:#2}}
\end{center}
}
\newcommand{\gnu}[2]{
\begin{figure}[hptb]
\begin{center}
\input{#2}
\caption{#1}
\label{fig:#2}
\end{center}
\end{figure}
}

\begin{document}
\title{応用プラズマ工学}
\author{82311971 佐々木良輔}
\date{}
\maketitle
\subsection*{(1)}
流体粒子の運動方程式において時間微分項を0とすると
\begin{align}
  {
    \everymath{\displaystyle}
    \begin{array}{cc}
      &0=ne_j{\bm E}-\nabla p_j-n\nu_jm_j{\bm u}_j\\
      \iff&n\nu_jm_j{\bm u}_j=ne_j{\bm E}-\nabla p_j\\
      \iff&{\bm u}_j=\frac{e_j{\bm E}}{m_j\nu_j}-\frac{\nabla p_j}{n\nu_jm_j}
    \end{array}
  }
\end{align}
を得る.
\subsection*{(2)}
(1)式において$p_j=nk_BT_j$とすると
\begin{align}
  {\bm u}_j=\frac{e_j{\bm E}}{m_j\nu_j}-\frac{\nabla(nk_BT_j)}{n\nu_jm_j}
\end{align}
$T_j$が空間的に一様であるならば
\begin{align}
  \begin{split}
    {\bm u}_j&=\frac{e_j{\bm E}}{m_j\nu_j}-\frac{k_BT_j}{\nu_jm_j}\frac{\nabla n}{n}\\
    &=:\mu_j{\bm E}-D_j\frac{\nabla n}{n}
  \end{split}
\end{align}
となる.
\subsection*{(3)}
$\mu_j$は一般的に移動度と呼ばれる.荷電粒子が物質中を移動すると,その速度は電場による加速と散乱によりある終端速度へ至る.
移動度は単位電場を印加した際の粒子の運動の終端速度となる.
$D_j$は拡散係数であり,粒子に大きさ1の濃度勾配が与えられた際に生じる粒子の速度となる.
\subsection*{(4)}
熱速度と温度の関係
\begin{align}
  \frac{1}{2}m_jv_{th,h}^2=\frac{1}{2}k_BT_j
\end{align}
から
\begin{align}
  v_{th,j}^2=\frac{k_BT_j}{m_j}
\end{align}
両辺を$\nu_j$で割ると
\begin{align}
  \frac{v_{th,j}^2}{\nu_j}=\frac{k_BT_j}{m_j\nu_j}=D_j
\end{align}
ここで平均自由行程$\lambda_j$には
\begin{align}
  \lambda_j=v_{th,h}\frac{1}{\nu_j}=v_{th,j}\tau_j
\end{align}
の関係があったので(6)式は
\begin{align}
    D_j&=\frac{\lambda_j^2}{\tau_j^2}\frac{1}{\nu_j}=\frac{\lambda_j^2}{\tau_j}
\end{align}
を得る.
\end{document}