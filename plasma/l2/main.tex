\documentclass[uplatex,a4j,11pt,dvipdfmx]{jsarticle}
\usepackage{listings,jvlisting}
\bibliographystyle{junsrt}

\usepackage{url}

\usepackage{graphicx}
\usepackage{gnuplot-lua-tikz}
\usepackage{pgfplots}
\usepackage{tikz}
\usepackage{amsmath,amsfonts,amssymb}
\usepackage{bm}
\usepackage{siunitx}

\makeatletter
\def\fgcaption{\def\@captype{figure}\caption}
\makeatother
\newcommand{\setsections}[3]{
\setcounter{section}{#1}
\setcounter{subsection}{#2}
\setcounter{subsubsection}{#3}
}
\newcommand{\mfig}[3][width=15cm]{
\begin{center}
\includegraphics[#1]{#2}
\fgcaption{#3 \label{fig:#2}}
\end{center}
}
\newcommand{\gnu}[2]{
\begin{figure}[hptb]
\begin{center}
\input{#2}
\caption{#1}
\label{fig:#2}
\end{center}
\end{figure}
}

\begin{document}
\title{応用プラズマ工学}
\author{佐々木良輔}
\date{}
\maketitle
\section{}
質量$m$,電荷$Ze$の粒子が磁束密度$B$中で行うサイクロトロン運動の半径$r$は
\begin{align}
  r&=\frac{mv}{ZeB}=\frac{1}{ZeB}\sqrt{2mk_BT}\ \left[\frac{\si{(\kilo\gram.\joule)^{1/2}}}{\si{\coulomb.\tesla}}=\si{\metre}\right]
\end{align}
ここで$e=1.602\times10^{-19}\ \si{\coulomb}$は電気素量, $k_B=1.602\times10^{-19}\ \si{\joule.\electronvolt^{-1}}$はボルツマン定数である.
水素負イオンにおいて,電荷は$Z=1$, 質量は水素の原子量が$1$であることからアボガドロ数$N_A$を用いて
\begin{align}
  \frac{1}{N_A}\times10^{-3}=1.660\times10^{-27}\ \si{\kilo\gram}
\end{align}
である.したがって$T=1\ \si{\electronvolt}$, $B=0.1\ \si{\milli\tesla}$のときのサイクロトロン半径は
\begin{align}
    r&=\frac{\sqrt{2\times1.660\times10^{-27}\times1.602\times10^{-19}\times1}}{1\times1.602\times10^{-19}\times0.1\times10^{-3}}=1.44\ \si{\metre}
\end{align}
$B=50\ \si{\milli\tesla}$のときは
\begin{align}
  r&=\frac{\sqrt{2\times1.660\times10^{-27}\times1.602\times10^{-19}\times1}}{1\times1.602\times10^{-19}\times50\times10^{-3}}=1.44\ \si{\metre}=2.88\times10^{-3}\ \si{\metre}
\end{align}
となる.
\newpage
\section{}
磁場が空間勾配を持つとき,正に帯電したイオンは${\bm B}\times\nabla {\bm B}$の方向に,負に帯電したイオンは逆方向にドリフトする.
したがって図\ref{fig:q2.png}のように$+x$方向の磁場を$-z$方向に勾配させたとき,水素(正)イオンは$+y$方向(紙面奥)へドリフトし,
水素負イオンは$-y$方向(紙面手前)にドリフトする.
\mfig[width=4cm]{q2.png}{磁場とその空間勾配}
\end{document}