\documentclass[uplatex,a4j,11pt,dvipdfmx]{jsarticle}
\usepackage{listings,jvlisting}
\bibliographystyle{junsrt}

\usepackage{url}

\usepackage{graphicx}
\usepackage{gnuplot-lua-tikz}
\usepackage{pgfplots}
\usepackage{tikz}
\usepackage{amsmath,amsfonts,amssymb}
\usepackage{bm}
\usepackage{siunitx}

\makeatletter
\def\fgcaption{\def\@captype{figure}\caption}
\makeatother
\newcommand{\setsections}[3]{
\setcounter{section}{#1}
\setcounter{subsection}{#2}
\setcounter{subsubsection}{#3}
}
\newcommand{\mfig}[3][width=15cm]{
\begin{center}
\includegraphics[#1]{#2}
\fgcaption{#3 \label{fig:#2}}
\end{center}
}
\newcommand{\gnu}[2]{
\begin{figure}[hptb]
\begin{center}
\input{#2}
\caption{#1}
\label{fig:#2}
\end{center}
\end{figure}
}

\begin{document}
\title{応用プラズマ工学}
\author{82311971 佐々木良輔}
\date{}
\maketitle
\section*{問1}
電荷$-e$を持った電子が電場$E(x)$から受ける力は$-eE(x)$なので,数密度が$n(x)$の電子が受ける単位体積あたりの力は
\begin{align}
  -eE(x)n(x)
\end{align}
である.
\section*{問2}
底面積$S$, 高さ$\Delta x$の円柱の体積は$S\Delta x$である.
$\Delta x$が十分小さく数密度$n(x)$の変化を無視できるとき,この柱に含まれる電子の数は
\begin{align}
  n(x)S\Delta x
\end{align}
であり,これが電場$E(x)$から受ける力は
\begin{align}
  -eE(x)n(x)S\Delta x
\end{align}
である.
\section*{問3}
微小円柱要素の両端における圧力は$p(x)$, $p(x+\Delta x)$であるので,
この円柱内に含まれる電子に関する運動方程式は電子の質量$m_e$を用いて
\begin{align}
  \begin{split}
    n(x)m_e\ddot{x}&=p(x)S-p(x+\Delta x)S-eE(x)n(x)S\Delta x\\
    &=-\Delta pS-eE(x)n(x)S\Delta x
  \end{split}
\end{align}
ここで電子の質量が小さいことから$m_e\rightarrow0$とすると
\begin{align}
  \begin{array}{cl}
    &0=-\Delta pS-eE(x)n(x)S\Delta x\\
    \iff&0=-\cfrac{\Delta p}{\Delta x}-eE(x)n(x)
  \end{array}
\end{align}
\section*{問4}
$\Delta x\rightarrow0$の極限において(5)式は
\begin{align}
  \cfrac{dp}{dx}=-eE(x)n(x)
\end{align}
となる.
ここで理想気体の状態方程式$p=nkT$, 電場と電位の関係$E=-\nabla\phi$より(6)は
\begin{align}
  \begin{array}{crcl}
    &\cfrac{d}{dx}(n(x)kT)&=&-en(x)\left(-\cfrac{d\phi}{dx}\right)\\
    \iff&\cfrac{\frac{dn}{dx}}{n}&=&\cfrac{e}{kT}\cfrac{d\phi}{dx}
  \end{array}
\end{align}
ここで解として
\begin{align}
  n(x)=Ae^{f(x)}
\end{align}
という形を仮定する.ただし$A$は定数である. (7)から
\begin{align}
  \begin{array}{crcl}
    &\cfrac{n_0e^{f(x)}\frac{df}{dx}}{n_0e^{f(x)}}&=&\cfrac{e}{kT}\cfrac{d\phi}{dx}\\
    \iff&\cfrac{df}{dx}&=&\cfrac{e}{kT}\cfrac{d\phi}{dx}\\
    \iff&f(x)&=&\cfrac{e}{kT}\left(\phi(x)-B\right)\\
  \end{array}
\end{align}
ただし$B$は定数である
したがって電子密度$n(x)$は
\begin{align}
  n(x)=A\exp\left(\frac{e}{kT}\left(\phi(x)-B\right)\right)
\end{align}
と表される.さらに$x=0$において$\phi(0)=0$から
\begin{align}
  n(0)=A\exp\left(\frac{e}{kT}\left(0-B\right)\right)
\end{align}
なので
\begin{align}
  n(x)=n(0)\exp\left(\frac{e}{kT}\phi(x)\right)
\end{align}
となる.
\end{document}