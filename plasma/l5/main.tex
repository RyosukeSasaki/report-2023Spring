\documentclass[uplatex,a4j,11pt,dvipdfmx]{jsarticle}
\usepackage{listings,jvlisting}
\bibliographystyle{junsrt}

\usepackage{url}

\usepackage{graphicx}
\usepackage{gnuplot-lua-tikz}
\usepackage{pgfplots}
\usepackage{tikz}
\usepackage{amsmath,amsfonts,amssymb}
\usepackage{bm}
\usepackage{siunitx}

\makeatletter
\def\fgcaption{\def\@captype{figure}\caption}
\makeatother
\newcommand{\setsections}[3]{
\setcounter{section}{#1}
\setcounter{subsection}{#2}
\setcounter{subsubsection}{#3}
}
\newcommand{\mfig}[3][width=15cm]{
\begin{center}
\includegraphics[#1]{#2}
\fgcaption{#3 \label{fig:#2}}
\end{center}
}
\newcommand{\gnu}[2]{
\begin{figure}[hptb]
\begin{center}
\input{#2}
\caption{#1}
\label{fig:#2}
\end{center}
\end{figure}
}

\begin{document}
\title{応用プラズマ工学}
\author{82311971 佐々木良輔}
\date{}
\maketitle
$x$方向の速さが$v_x$の粒子密度$dn_x$は,粒子の速度分布がMaxwell分布に従うことから
\begin{align}
  \begin{split}
    dn_x&=\int_{-\infty}^{\infty}dv_y\int_{-\infty}^{\infty}dv_z\ n_e\left(\frac{m_e}{2\pi k_BT}\right)^{3/2}\exp\left(-\frac{m_e}{2k_BT}(v_x^2+v_y^2+v_z^2)\right)\\
    &=n_e\left(\frac{m_e}{2\pi k_BT}\right)^{3/2}\exp\left(-\frac{m_ev_x^2}{2k_BT}\right)\left(\int_{-\infty}^{\infty}dv\ \exp\left(-\frac{m_ev^2}{2k_BT}\right)\right)^2
  \end{split}
\end{align}
ここでガウス積分$\int_{-\infty}^{\infty}dx\ e^{-ax^2}=\sqrt{\pi/a}$を用いて
\begin{align}
  \begin{split}
    dn_x&=n_e\left(\frac{m_e}{2\pi k_BT}\right)^{3/2}\exp\left(-\frac{m_ev_x^2}{2k_BT}\right)\left(\pi\frac{2k_BT}{m_e}\right)\\
    &=n_e\sqrt{\frac{m_e}{2\pi k_BT}}\exp\left(-\frac{m_ev_x^2}{2k_BT}\right)
  \end{split}
\end{align}
である.ここで$y-z$面内にある単位面積の領域を単位時間に通過する速さ$v_x$の粒子数は,
図の体積に含まれる粒子数に等しい.その値は
\begin{align}
  1\times v_x\times dn_x=v_xdn_x
\end{align}
したがってこの領域を単位時間に右向きに通過する全粒子数は,これを$[0,\infty)$で積分すればよい.
ガウス積分$\int_{0}^{\infty}dx\ xe^{-ax^2}=1/2a$より
\begin{align}
  \begin{split}
    \int_{0}^{\infty}\ v_xdn_x&=
    n_e\sqrt{\frac{m_e}{2\pi k_BT}}\int_{0}^{\infty}dv_x\ v_x\exp\left(-\frac{m_ev_x^2}{2k_BT}\right)\\&=
    n_e\sqrt{\frac{m_e}{2\pi k_BT}}\frac{1}{2}\frac{2k_BT}{m_e}\\
    &=n_e\sqrt{\frac{k_BT}{2\pi m_e}}\\
    &=\frac{1}{4}n_e\sqrt{\frac{8k_BT}{\pi m_e}}=\frac{1}{4}n_e\overline{c_e}
  \end{split}
\end{align}
となる.
\bibliography{ref.bib}
\end{document}