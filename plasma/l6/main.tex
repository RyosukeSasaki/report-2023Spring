\documentclass[uplatex,a4j,11pt,dvipdfmx]{jsarticle}
\usepackage{listings,jvlisting}
\bibliographystyle{junsrt}

\usepackage{url}

\usepackage{graphicx}
\usepackage{gnuplot-lua-tikz}
\usepackage{pgfplots}
\usepackage{tikz}
\usepackage{amsmath,amsfonts,amssymb}
\usepackage{bm}
\usepackage{siunitx}

\makeatletter
\def\fgcaption{\def\@captype{figure}\caption}
\makeatother
\newcommand{\setsections}[3]{
\setcounter{section}{#1}
\setcounter{subsection}{#2}
\setcounter{subsubsection}{#3}
}
\newcommand{\mfig}[3][width=15cm]{
\begin{center}
\includegraphics[#1]{#2}
\fgcaption{#3 \label{fig:#2}}
\end{center}
}
\newcommand{\gnu}[2]{
\begin{figure}[hptb]
\begin{center}
\input{#2}
\caption{#1}
\label{fig:#2}
\end{center}
\end{figure}
}

\begin{document}
\title{応用プラズマ工学}
\author{82311971 佐々木良輔}
\date{}
\maketitle
密度連続の式
\begin{align}
  \frac{\partial n}{\partial t}+{\rm div}(n\vec{v})=S_i
\end{align}
において両辺をシース内側の領域$V$で体積平均を行う.
\begin{align}
  \frac{1}{V}\int_VdV\frac{\partial n}{\partial t}+\frac{1}{V}\int_VdV{\rm div}(n\vec{v})=\frac{1}{V}\int_VdV\ S_i
\end{align}
左辺第1項の微分と積分を交換すると
\begin{align}
  \frac{1}{V}\int_VdV\frac{\partial n}{\partial t}=\frac{\partial}{\partial t}\frac{1}{V}\int_VdV\ n=\frac{\partial\overline{n}}{\partial t}
\end{align}
ここで$\overline{n}$はもはや空間に依存しないため
\begin{align}
  \frac{\partial\overline{n}}{\partial t}=\frac{d\overline{n}}{dt}
\end{align}
また左辺第2項において,発散定理を用いて
\begin{align}
  \frac{1}{V}\int_VdV{\rm div}(n\vec{v})=\frac{1}{V}\int_{\partial V}dS(n\vec{v})\cdot{\bm e}_n
\end{align}
ただし$\partial V$は$V$の表面, ${\bm e}_n$はシース表面での法線ベクトルとした.
シース表面で$n=n_{is}$, $\vec{v}=v_{is}{\bm e}_n$の一定値を取るとする.
ただしシース入口においてはイオンの速度がイオン音速$C_s$に一致することから, $v_{is}=C_s$である.
(5)式はシース表面積$S$を用いて
\begin{align}
  \begin{split}
    \frac{1}{V}\int_{\partial V}dS(n\vec{v})\cdot{\bm e}_n&=\frac{S}{V}n_{is}C_s\\
    &=\frac{S}{V}C_s\frac{n_{is}}{\overline{n}}\overline{n}\\
    &=\frac{\overline{n}}{\frac{V}{S\alpha C_s}}=\frac{\overline{n}}{\tau_p}
  \end{split}
\end{align}
となる.
最後に右辺第1項は
\begin{align}
  \frac{1}{V}\int_VdV\ S_i=\overline{S}_i
\end{align}
である.以上から(2)式は
\begin{align}
  \begin{array}{cc}
    &\cfrac{d\overline{n}}{dt}+\cfrac{\overline{n}}{\tau_p}=\overline{S}_i\\
    \iff&\cfrac{d\overline{n}}{dt}=\overline{S}_i-\cfrac{\overline{n}}{\tau_p}
  \end{array}
\end{align}
となり,粒子バランスの式が得られた.ここで$\tau_p$は
\begin{align}
  \tau_p=\frac{V}{S}\frac{1}{\alpha C_s}
\end{align}
であった.プラズマ容器として一辺の長さが$L$の立方体を仮定すると
\begin{align}
  \frac{V}{S}\propto L
\end{align}
またシース表面での粒子束密度は$\Gamma_s=n_{is}C_s$であり,これは定常状態において壁への粒子束に等しい.
したがって
\begin{align}
  \tau_p=L\frac{1}{\frac{1}{\overline{n}}\Gamma_s}=\frac{L\overline{n}}{\Gamma_s}
\end{align}
ここで$L\overline{n}$は,長さ$L$,断面積$1$の空間に存在する粒子数と考えられる.
したがって$L\overline{n}/\Gamma_s$は,この空間に存在する粒子がシース表面での粒子束密度で失われるのに要する時間であり,
生成されたプラズマの寿命と考えることができる.
\bibliography{ref.bib}
\end{document}